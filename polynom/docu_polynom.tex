\documentclass[fragile=singleslide]{beamer}

\usetheme{polynom}
\usefonttheme[onlymath]{serif}
\boldmath{}

% Please refer to the manual of minted, if you want to compile this
% documentation from source and have to install it!
% https://github.com/gpoore/minted/blob/master/source/minted.pdf 
% \usepackage{minted}
\usepackage{tgadventor}

\title{Polynom}
\subtitle{A modern, clean 16:9 beamer template}
\author{Alexander Bartolomey, Adrian C. Hinrichs}
\date{\today}

\begin{document}

\setbeamertemplate{title page}[polynom]{}
\begin{frame}
  \titlepage
\end{frame}

\begin{frame}{Table of Contents}
  \tableofcontents
\end{frame}


\section{Motivation}
\begin{frame}
  This theme is developed to match PowerPoint-Templates by Alexander
  Bartolomey GitLab / GitHub.

  It is developed because of the lack of clean LaTeX-Templates that
  are designed with 16:9 in mind.  (Also I did not want to use the
  theme provided by my university)
\end{frame}

\begin{frame}{Preamble}
  I highly recommend the use of XeLaTeX, especially the unicode and
  font support are very beneficial for the creation of presentations.

  Nevertheless, we will try to keep the theme working on standard
  \LaTeX, but the theme will probably be less tested on \LaTeX
\end{frame}

\section{Features}
\begin{frame}[fragile=singleslide]
  \framesubtitle{Clean and pleasing:}
  \frametitle{Framedesign}
  To use stacked headlines, you have to specify
  \begin{itemize}
  \item the upper line with \verb|\framesubtitle{...}|
  \item the lowe (main) line with \verb|\frametitle{...}|
  \end{itemize}
\end{frame}

\begin{frame}{Math in Polynom}
  Looks very nice, thanks to serifs and boldfaced letters
   \[ A=U\Sigma V^T\]
   for every matrix \( A\in \mathbb{R}^{m\times n}\) with
   \[\Sigma:=\mathrm{diag}(\sigma_1,\dots,\sigma_p) \in \mathbb{R}^{m\times n}, p = \min(m,n)\]
\end{frame}

\subsection{Section Titles}
\begin{frame}[fragile=singleslide]{Section Titles}
  \begin{itemize}
  \item Activated by default
    \begin{itemize}
    \item deactivated by calling polynom with \verb|\usetheme[sectiontitles=f]{polynom}|
    \end{itemize}
  \item For Sections, Subsections and Subsubsections
  \end{itemize}
\end{frame}

\begin{frame}[fragile=singleslide]
  \frametitle{Images on section titles}
  \framesubtitle{Special Feature}
  To add an image to the title of the next section, call
  \verb|\nextsectionimage{#1}|, where
  \verb|#1| is the image file, as you would specify it
  for \verb|\includegraphics|. Keep in mind, that the image height
  \alert{always} will be matched to the height of the bar, which has
  an aspect-ratio of 8:3.\\

  See next slide for an example (Photo by \href{https://unsplash.com/photos/NRQV-hBF10M}{Bailey Zindel} on \href{https://unsplash.com/}{Unsplash})
\end{frame}

\nextsectionimage{docu_landscape}
\section{Font Selection}
\begin{frame}[fragile=singleslide]{Main Font}
  As a main font, we advise the usage of a heavy bold-face geometric typeface.  
  We formerly used Googles Product Sans, but since it is not available for free
  due to licensing, there are several alternatives:  
  \begin{itemize}
  \item \href{http://www.tug.dk/FontCatalogue/texgyreadventor/}{\TeX
    Gyre Adventor} (Free) (Works for \LaTeX and Xe\LaTeX, used in this document)
  \item Futura PT Heavy
  \item Helvetica, using \verb|\usepackage{helvet}|
  \item \href{https://www.1001fonts.com/gillius-adf-font.html}{Gillius
      ADF} (Free)
  \item \href{https://fonts.google.com/specimen/Karla}{Karla} (Free)
  \end{itemize}
\end{frame}

\begin{frame}{Monospaced Font}
  Any monospaced font you like, we prefer
  \href{https://mozilla.github.io/Fira/}{Fira Mono}. Please do not use
  \href{https://github.com/tonsky/FiraCode}{Fira Code} or other fonts with 
  programming ligatures for your presentations because viewers who do not 
  know them can (and will!) get confused by XeLaTeX ligature support, e.g.
  condensing !== to three horizontal lines with one skewed vertical right 
  through it.
\end{frame}
 
\begin{frame}{Textsizes}
  
  {\tiny tiny}
  {\scriptsize scriptsize}
  {\footnotesize footnotesize}
  {\small small}
  {\normalsize normalsize} (Default)
  {\large large}
  {\Large Large}
  {\LARGE LARGE}
  {\huge huge}
  {\Huge Huge}
\end{frame}

\section{Color Palette}

\newcommand{\clrbx}[1]{
  \begin{beamercolorbox}[wd=5em,ht=5ex,dp=1.125ex,center]{#1}
    \small#1
  \end{beamercolorbox}}
\newcommand{\paletteColors}{
  \clrbx{palette primary}
  \clrbx{palette secondary}
  \clrbx{palette tertiary}
  \clrbx{palette quaternary}}


\begin{frame}{Default Color Theme}
  The default palette consists of the following colors:\\
  \paletteColors
\end{frame}

\mode<presentation>

\definecolor{polynom-blue-lighter}{RGB}{177 198 255}
\definecolor{polynom-blue-light}{RGB}  {101 142 255}
\definecolor{polynom-blue}{RGB}        {41 98 255}
\definecolor{polynom-blue-dark}{RGB}   {24 57 148}
\definecolor{polynom-blue-darker}{RGB} {15 37 96}

\definecolor{polynom-space-grey}{RGB}{51 51 51}
\definecolor{polynom-white}{RGB}{255 255 255}

\definecolor{polynom-red-lighter}{RGB}  {255 176 161}
\definecolor{polynom-red-light}{RGB}{255 112 85}
\definecolor{polynom-red}{RGB}        {255 62 25}
\definecolor{polynom-red-dark}{RGB}   {148 36 15}
\definecolor{polynom-red-darker}{RGB} {96 23 9}

\definecolor{polynom-green-lighter}{RGB}{179 204 129}
\definecolor{polynom-green-light}{RGB}  {159 204 68}
\definecolor{polynom-green}{RGB}        {143 204 20}
\definecolor{polynom-green-dark}{RGB}   {68 97 10}
\definecolor{polynom-green-darker}{RGB} {32 45 4}

\definecolor{black}   {RGB}{  0   0   0}


\setbeamercolor*{normal text}{fg=polynom-space-grey}



\newcommand{\setPaletteBlue}{
\setbeamercolor*{structure}{fg=polynom-blue}
\setbeamercolor{palette primary}{   fg=polynom-white,bg=polynom-blue}
\setbeamercolor{palette secondary}{ fg=polynom-white,bg=polynom-blue-dark}
\setbeamercolor{palette tertiary}{  fg=polynom-white,bg=polynom-blue-darker}
\setbeamercolor{palette
  quaternary}{fg=polynom-white,bg=polynom-blue-darker}
}
\newcommand{\setPaletteRed}{
\setbeamercolor*{structure}{fg=polynom-red}
\setbeamercolor{palette primary}{   fg=polynom-white,bg=polynom-red}
\setbeamercolor{palette secondary}{ fg=polynom-white,bg=polynom-red-dark}
\setbeamercolor{palette tertiary}{  fg=polynom-white,bg=polynom-red-darker}
\setbeamercolor{palette
  quaternary}{fg=polynom-white,bg=polynom-red-darker}
}

\newcommand{\setPaletteGreen}{
\setbeamercolor*{structure}{fg=polynom-green}
\setbeamercolor{palette primary}{   fg=polynom-white,bg=polynom-green}
\setbeamercolor{palette secondary}{ fg=polynom-white,bg=polynom-green-dark}
\setbeamercolor{palette tertiary}{  fg=polynom-white,bg=polynom-green-darker}
\setbeamercolor{palette
  quaternary}{fg=polynom-white,bg=polynom-green-darker}
}
\setPaletteBlue
% Sidebar
\setbeamercolor*{sidebar}{parent=palette primary}
\setbeamercolor*{palette sidebar primary}{use=normal text,fg=normal text.fg}
\setbeamercolor*{palette sidebar secondary}{use=normal text,fg=normal text.fg}
\setbeamercolor*{palette sidebar tertiary}{use=normal text,fg=normal text.fg}
\setbeamercolor*{palette sidebar quaternary}{use=normal text,fg=normal text.fg}

% Outer structure
\setbeamercolor{frametitle}{fg=polynom-space-grey}
\setbeamercolor{title}{fg=polynom-space-grey}

\setbeamercolor{subtitle}{fg=polynom-space-grey}
\setbeamercolor{author}{fg=polynom-space-grey}

\setbeamercolor{date}{fg=polynom-space-grey}
\setbeamercolor{section}{fg=polynom-space-grey}
\setbeamercolor{subsection}{fg=polynom-space-grey}
\setbeamercolor{subsubsection}{fg=polynom-space-grey}
% Blocks and special text
\setbeamercolor*{block title}{fg=polynom-blue-dark, bg=polynom-blue-light}
\setbeamercolor{block body}{bg=polynom-blue-lighter}

\setbeamercolor*{example text}{fg=polynom-green-dark}
\setbeamercolor{block title example}{bg=polynom-green-light}
\setbeamercolor{block body example}{bg=polynom-green-lighter}

\setbeamercolor*{alerted text}{fg=polynom-red}
\setbeamercolor{block title alerted}{fg=polynom-red-dark, bg=polynom-red-light}
\setbeamercolor{block body alerted}{bg=polynom-red-lighter}

% banner page 
\setbeamercolor{banner page}{bg=polynom-white, fg=polynom-space-grey}
\setbeamercolor{banner page invert}{bg=polynom-space-grey, fg=white}


\setbeamercolor{footline}{bg=polynom-white, fg=polynom-space-grey}
\setbeamercolor{footline invert}{bg=polynom-space-grey,
  fg=polynom-white}
\setbeamercolor{framenumber}{use=palette primary,bg=palette primary.bg,fg=polynom-white}
\mode<all>


\begin{frame}[fragile=singleslide]{Color Theme \texttt{polynomseconddegree}}
  Load it via \verb|\usebeamercolortheme{polynomseconddegree}|. 
  The main palette consists of the following colors:\\
  \paletteColors
  
  \texttt{polynomseconddegree} provides additional colors schemes,
  which you can switch dynamically, the default one can be restored by by
  \verb|\setPaletteBlue|
\end{frame}

\setPaletteRed
\begin{frame}[fragile=singleslide]{Color Theme \texttt{polynomseconddegree}}
  \framesubtitle{Red}

  To use the red color scheme, call \verb|\setPaletteRed|
  before the first frame you want to use it on (also affects
  section-titles etc.)\\
  \paletteColors
  
\end{frame}

\setPaletteGreen
\begin{frame}[fragile=singleslide]{Color Theme \texttt{polynomseconddegree}}
  \framesubtitle{Green}

  To use the green color scheme, call \verb|\setPaletteGreen|
  before the first frame you want to use it on (also affects
  section-titles etc.) \\
  \paletteColors
  
\end{frame}
\setPaletteBlue

\section{Banner-Pages}
\begin{frame}[fragile=singleslide]{Banner Pages}
  To highlight important messages, you can create banner pages with the
  following code:
  \verb|\setbeamertemplate{banner page}[polynom][Bannerpagetext]|
  Now you have to use this on your next frame:
  \verb|\usebeamertemplate{banner page}|

  Alternatively, you can use the command
  \verb|\bannerpage{...}|
\end{frame}

\setbeamertemplate{banner page}[polynom][buzzword!]
\begin{frame}
  \usebeamertemplate{banner page}
\end{frame}

\begin{frame}[fragile=singleslide]
  If you replace \verb|banner page| by
  \verb|banner page invert|, or use \verb|\bannerpageinvert{...}|
  you get an...
\end{frame}

\bannerpageinvert{inverted banner page!}


\section{Postamble}
\begin{frame}{Development}
  This theme is under active development to progressively match the
  PowerPoint template so that the PowerPoint and LaTeX--template will
  be usable in a similar (even though not completely identical (eyes on
  ligature--support and animations) manner.

  The development takes place in
  \href{https://git.rwth-aachen.de/ACHinrichs/LaTeX-templates/}{\texttt{this}}
  repository, please submit any bug reports or feature requests there.
\end{frame}

\begin{frame}{Thanks}
  Many thanks to \href{https://www.occloxium.com/}{Alexander Bartolomey} for 
  the great PowerPoint template and his work on the LaTeX implementation. 
\end{frame}
\end{document}

