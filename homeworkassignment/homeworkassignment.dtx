% \iffalse meta-comment
%
% This File may be distributed and/or modified under the condition of the below
% license.
%
%
% MIT License
%
%
% Copyright (c) 2016-2019 by Adrian C. Hinrichs
%
%
%
% Permission is hereby granted, free of charge, to any person obtaining a copy
% of this software and associated documentation files (the "Software"), to deal
% in the Software without restriction, including without limitation the rights
% to use, copy, modify, merge, publish, distribute, sublicense, and/or sell
% copies of the Software, and to permit persons to whom the Software is
% furnished to do so, subject to the following conditions:
%
% The above copyright notice and this permission notice shall be included in all
% copies or substantial portions of the Software.
%
% THE SOFTWARE IS PROVIDED "AS IS", WITHOUT WARRANTY OF ANY KIND, EXPRESS OR
% IMPLIED, INCLUDING BUT NOT LIMITED TO THE WARRANTIES OF MERCHANTABILITY,
% FITNESS FOR A PARTICULAR PURPOSE AND NONINFRINGEMENT. IN NO EVENT SHALL THE
% AUTHORS OR COPYRIGHT HOLDERS BE LIABLE FOR ANY CLAIM, DAMAGES OR OTHER
% LIABILITY, WHETHER IN AN ACTION OF CONTRACT, TORT OR OTHERWISE, ARISING FROM,
% OUT OF OR IN CONNECTION WITH THE SOFTWARE OR THE USE OR OTHER DEALINGS IN THE
% SOFTWARE.
% \fi
\def\fileversion{v3.2c}
\def\filedate{2019/04/16}
% \iffalse
%<*driver>
\ProvidesFile{homeworkassignment.dtx}
%</driver>
%<class>\NeedsTeXFormat{LaTeX2e}[2005/12/01]
%<class>\ProvidesClass{homeworkassignment}[\filedate\space\fileversion]
%<*driver>
\documentclass{ltxdoc}
\EnableCrossrefs
\CodelineIndex
\RecordChanges

% ATTENTION, DO NOT USE THIS SECTION FOR INCLUSION OF PACKAGES TO THE
% *CLASS*, THIS PLACE IS WHERE PACKAGES FOR THE *DOCUMENTATION* ARE LOADES

\usepackage[ngerman,english]{babel}
\usepackage[autostyle,german=guillemets]{csquotes}
\usepackage{hyperref}
\usepackage{amssymb}
\usepackage{amsmath}
\usepackage{longtable}
\usepackage{soul}
\usepackage{todonotes}
\usepackage[T1]{fontenc}
\usepackage{framed}

\DeclareMathOperator{\GL}{GL}
\DeclareMathOperator{\id}{id}
\DeclareMathOperator{\Var}{Var}
\DeclareMathOperator{\Perm}{Perm}
\DeclareMathOperator{\MComb}{MComb}
\DeclareMathOperator{\Comb}{Comb}
\DeclareMathOperator{\Pot}{Pot}
\DeclareMathOperator{\Map}{Map}
\DeclareMathOperator{\Hom}{Hom}
\DeclareMathOperator{\Ker}{Ker}
\DeclareMathOperator{\Intpol}{Intpol}
\DeclareMathOperator{\Pol}{Pol}
\DeclareMathOperator{\Sol}{Sol}
\DeclareMathOperator{\Bin}{Bin}
\DeclareMathOperator{\charakteristik}{char}
\DeclareMathOperator{\fo}{fo}
\DeclareMathOperator{\first}{fi}
\DeclareMathOperator{\la}{la}

\begin{document}
\DocInput{homeworkassignment.dtx}
\end{document}
%</driver>
% \fi
%
% \CheckSum{0}
%
% \CharacterTable {Upper-case
% \A\B\C\D\E\F\G\H\I\J\K\L\M\N\O\P\Q\R\S\T\U\V\W\X\Y\Z Lower-case
% \a\b\c\d\e\f\g\h\i\j\k\l\m\n\o\p\q\r\s\t\u\v\w\x\y\z Digits
% \0\1\2\3\4\5\6\7\8\9 Exclamation \!  Double quote \" Hash (number)
% \# Dollar \$ Percent \% Ampersand \& Acute accent \' Left paren
% \( Right paren \) Asterisk \* Plus \+ Comma \, Minus \- Point \.
% Solidus \/ Colon \: Semicolon \; Less than \< Equals \= Greater than
% \> Question mark \?
% Commercial at \@     Left bracket  \[     Backslash     \\
%   Right bracket \] Circumflex \^ Underscore \_ Grave accent \` Left
% brace \{ Vertical bar \| Right brace \} Tilde \~}
%
% \DoNotIndex{\newcommand,\newenvironment}
%
% \title{The \textsf{homeworkassignment}\footnote{The name was changed
% with version v3.0, to become compatible with CTANs guidlines and to
% maintain a degree of backwards compatibility.  The class was called
% \textsf{HomeworkAssignment} prior to v3.0} class\thanks{This document
% corresponds to \textsf{homeworkassignment}~\fileversion, dated~
% \filedate.}}
% \author{Adrian C Hinrichs \\
% \texttt{adrian.hinrichs@rwth-aachen.de}}
%
% \maketitle
% \tableofcontents
% \section{Abstract}
% This class provides a relative simple document--type for homework,
% mainly created for assignments at the University This class is
% inherited from \texttt{article}, it is not perfect, but I am trying
% my verry best.
% \section{Dependencies}
% \subsection{Mandatory Dependencies}
% This class is build uppon \textrm{article}, so of course the first
% dependency is:
% \newcommand{\dep}[4]{\item[\texttt{#1}]\textsc{#2}, \url{#3},\\#4}{}
% \begin{description}
% \dep{article}{1992 Leslie Lamport, 1994-97 Frank Mittelbach
% Johannes Braams, The \LaTeX-Team}{https://www.ctan.org/pkg/kvoptions}
% \end{description}
% Because I am very lazy, the \texttt{homeworkassignment} is \enquote{a little
% bit} bloated. These are all required packages:
% \begin{description}
% \dep{kvoptions}{Heiko
% Oberdiek}{https://www.ctan.org/pkg/kvoptions}{for
% \texttt{key=value}--style options}
% \dep{suffix}{David Kastrup}{https://www.ctan.org/pkg/suffix}{Makes
% it easy to define \texttt{$\backslash$macro*} commands}
% \dep{xifthen}{Josselin Noirel}{https://www.ctan.org/pkg/xifthen}{For
% if--else--structures}
% \dep{translations}{Clemens
% Niederberger}{https://www.ctan.org/pkg/translations}{Implements an
% easy method of translations.}
% \dep{amsmath}{The \LaTeX--Team, Frank Mittelbach
% Rainer Sch\"opf, et al.}{https://www.ctan.org/pkg/amsmath}{For
% better math-typesetting}
% \dep{amssymb}{American Mathematical
% Society}{mirror.ctan.org/fonts/amsfonts/doc/amssymb.pdf}{For more
% mathematical symbols}
% \dep{etoolbox}{Philipp Lehman (inactive), Joseph
% Wright}{https://www.ctan.org/pkg/etoolbox}{The package is a toolbox
% of programming facilities geared primarily towards \LaTeX class and
% package authors}
% \dep{array}{Frank Mittelbach, David Carlisle, The
% \LaTeX--Team}{https://www.ctan.org/pkg/array}{A new implementations
% for tables and arrays\todo{\texttt{array} possibly can be removed}}
% \dep{xparse}{Frank Mittelbach, Chris Rowley, David Carlisle, The
% \LaTeX3 Project}{https://ctan.org/pkg/xparse}{The package provides a
% high-level interface for producing documentlevel commands. In that
% way, it offers a replacement for \LaTeXe 's
% \texttt{\textbackslash{}newcommand} macro, with significantly improved
% functionality.}
% \dep{gillius}{Bob Tennent}{https://ctan.org/pkg/gillius}{A Gillian
% Sans inspired font, used for all sans serifes fonts}
% \dep{hyperref}{https://ctan.org/pkg/hyperref}{Sebastian Rahtz, Heiko
% Oberdiek}{For hyperrefs, obviously}
% \dep{xcolor}{Dr. Uwe Kern}{https://www.ctan.org/pkg/xcolor}{For
% coloring of ToDos}
% \end{description}
% \subsection{Recommended Dependencies}
% These are not loaded automatically, but require a switch as option
% (see \autoref{sec:Options}). The switch is typically the name of the
% package.
% \begin{description}
% \dep{tikz}{Till Tantau, Mark Wibrow, Christian Feuers\"anger et
% al.}{https://www.ctan.org/pkg/pgf}{An incredible powerfull image
% tool. When loading TikZ, the homeworkassignment automatically loads
% a shipload of TikZ--librarys and own styles\todo{I intend to move
% these styles to a own package, so that they are usable without the
% homeworkassignment}. See \autoref{imp:tikz} for more informations}
% \dep{listings}{Carsten Heinz, Brooks Moses, Jobst
% Hoffmann}{https://www.ctan.org/pkg/listings}{For
% source--code. Sourcecode in the homeworkassignment is automatically
% framed, printed in \texttt{scriptsize}, and linebeals will be
% introduced}
% \end{description}
% Loads required Packages
%    \begin{macrocode}
\RequirePackage{suffix}
\RequirePackage{fancyhdr}
\RequirePackage{xifthen}
\RequirePackage{translations}
\PassOptionsToPackage{fleqn}{amsmath}
\RequirePackage{amsmath}
\RequirePackage{amssymb}
\RequirePackage{etoolbox}
\RequirePackage{array}
\RequirePackage{xparse}
\RequirePackage{ifxetex}

\RequirePackage{wasysym}
\RequirePackage{adjustbox}

\RequirePackage{eso-pic}

\RequirePackage{xcolor}
%    \end{macrocode}
% \section{Options\label{sec:Options}}
% KV-Options is essential for this.
%    \begin{macrocode}
\RequirePackage{kvoptions} 
\SetupKeyvalOptions{ family=hwa,
  prefix=hwa@ }
\DeclareDefaultOption{\PassOptionsToClass{\CurrentOptionKey}{article}}
%    \end{macrocode}
% \DescribeMacro{problemstyle=<1>} \DescribeMacro{subproblemstyle=<1>}
% \DescribeMacro{subsubproblemstyle=<1>} These options allow the
% customizatuion of the displayed numbers.  For Example, if
% \texttt{problemstyle=Roman, subproblemstyle=arabic,
% subsubproblemstyle=roman} is passed, The first subsubproblem of the
% first subproblem
% of the first problem would be labled as \textbf{i)} of \textbf{Problem I.1}.\\
% Available options are \texttt{arabic}, \texttt{Alph}, \texttt{alph},
% \texttt{Roman}, and \texttt{roman}. Standard values are:
% \texttt{problemstyle=arabic, subproblemstyle=alph,
% subsubproblemstyle=roman}.
%    \begin{macrocode}
\DeclareStringOption[arabic]{problemsty}
\DeclareStringOption[alph]{subproblemsty}
\DeclareStringOption[roman]{subsubproblemsty}
%    \end{macrocode}
% \DescribeMacro{tikz} Loads TikZ-Package and a couple of Styles,
% usefull for Papers in Computer-Science and mathematics. See
% \ref{imp:tikz} for more informations
%    \begin{macrocode}
\DeclareBoolOption[false]{tikz}
%    \end{macrocode}
%
% \DescribeMacro{listings} Loads Listings Package and sets
% listing-layout to use a small fontsize. Adds indication for linebreaks.
%    \begin{macrocode}
\DeclareBoolOption[false]{listings}
%    \end{macrocode}
% \DescribeMacro{oneside, twoside}
% Changes layout. \textrm{oneside} is the complementary option to
% \texttt{twoside}\\
% Standard layout is twopaged.
%    \begin{macrocode}
\DeclareBoolOption[true]{twoside}
\DeclareComplementaryOption{oneside}{twoside}
%    \end{macrocode}
% \DescribeMacro{onecolumn,twocolumn}
% Changes layout. \textrm{onecolumn} is the complementary option to
% \texttt{twocolumn}.\\
% Standard Layout has one columns
%    \begin{macrocode}
\DeclareBoolOption[false]{twocolumn}
\DeclareComplementaryOption{onecolumn}{twocolumn}
%    \end{macrocode}
% \DescribeMacro{punchmark}
% Adds a mark for an hole puncher. 
% Standard Layout has no marking.
%    \begin{macrocode}
\DeclareBoolOption[false]{punchmark}
%    \end{macrocode}
% \DescribeMacro{hlines=<1>}
% KeyValue-option. Takes the level of hlines. Available are
% \texttt{all},\texttt{decreased},\texttt{header}, \texttt{none}, with
% decreasing number of lines; none displays none, header only the one
% under headers and decreased adds the big line in the title, while
% all displays all.
%    \begin{macrocode}
\DeclareStringOption[all]{hlines}
%    \end{macrocode}
% \DescribeMacro{todos=<1>}
% KeyValue-option. Takes which ToDos shall be displayed.  Available are
% \texttt{all} (default),\texttt{nolist},\texttt{none}.  See
% \autoref{cmd:TODOs} for explanation of the levels.
%    \begin{macrocode}
\DeclareStringOption[all]{todos}
%    \end{macrocode}
% \DescribeMacro{unicode-math} Loads the unicode--math--package and
% overwrites the damn \texttt{\textbackslash QED}--Command unicode--math
% introduces, that creates a filled out box and only works in
% math--mode, but not telling you that it only works in math--mode or
% overwrites an already existing command.  For a reason, that
% currently (06\textsuperscript{th} of December 2018) slips my mind
% completly, \texttt{unicode-math} needs to be loaded after
% \texttt{article}, because it needs \texttt{\normalsize} to be
% defined\\
%   \begin{framed}
%     \textsf{ATTENTION: Please do never, never, never, never, never
%     ever load unicode-math your self, because this breaks
%     \textbf{everything}\footnote{not eveything, but at least \texttt{\textbackslash{}QED}}}
%   \end{framed}
% 
% \texttt{\textbackslash{}end\{rant\}}\\
%
% If XeTeX is used, the default option for this is
% \texttt{true}, otherwise it is false.
%
%
% For the handling of the option, see \ref{QED}
%    \begin{macrocode}
\ifxetex
\DeclareBoolOption[true]{unicodemath}
\else
\DeclareBoolOption[false]{unicodemath}
\fi
%    \end{macrocode}
%
% Loads article and processes the options
%    \begin{macrocode}
\ProcessKeyvalOptions*
\ifhwa@twoside
\PassOptionsToClass{twoside}{article}
\else
\PassOptionsToClass{oneside}{article}
\fi
\ifhwa@twocolumn
\PassOptionsToClass{twocolumn}{article}
\else
\PassOptionsToClass{onecolumn}{article}
\fi
\LoadClass{article}

\newboolean{hwa@todos@inplace}
\newboolean{hwa@todos@list}
\setboolean{hwa@todos@inplace}{true}
\setboolean{hwa@todos@list}{true}
\ifthenelse{\equal{\hwa@todos}{all}}{
}{
  \ifthenelse{\equal{\hwa@todos}{nolist}}{
      \ClassWarning{homeworkassignment}{You specified todos=none,
        there will be no list of TODO}
      \setboolean{hwa@todos@list}{false}
  }{
    \ifthenelse{\equal{\hwa@todos}{none}}{
      \ClassWarning{homeworkassignment}{You specified todos=none,
        there will be no TODOs printed in the final document}
      \setboolean{hwa@todos@list}{false}
      \setboolean{hwa@todos@inplace}{false}
    }{
      \ClassError{homeworkassignment}{\hwa@todos is not a valid value
        for the option `todos`}
    }
  }
}
%    \end{macrocode}
% Load Hyperref (breaks if it is loaded before article
%    \begin{macrocode}
\RequirePackage{hyperref}
%    \end{macrocode}
% Loads listings, if wanted
%    \begin{macrocode}
\ifhwa@listings
\RequirePackage{listings}
\lstset{
  frame = single,
  breaklines = true,
  postbreak=\raisebox{0ex}[0ex][0ex]{\ensuremath{\hookrightarrow\space}},
  basicstyle=\scriptsize
}
\else
\empty
\fi
%    \end{macrocode}
% \begin{macro}{\hwa@hline@L...}
%   Defines new commands to output desired lines and change the
%   constant \textrm{\textbackslash{}hwa@headrulewidth}\\
%   \begin{framed}
%     \textsf{ATTENTION: \textrm{\textbackslash{}hwa@hline@LONE} breaks the
%     line automatically, in opposite to \textrm{\textbackslash{}hwa@hline@LTWO}}
%   \end{framed}
%    \begin{macrocode}

\newcommand{\hwa@hline@LONE}{\vspace{.25cm} {\hrule height 2pt}
  \vspace{.25cm}}
\newcommand{\hwa@hline@LTWO}{\vspace{.5cm} \hrule \vspace{.25cm}}
\newcommand{\hwa@headrulewidth}{.7pt}
\ifthenelse{\equal{\hwa@hlines}{all}}{
  \renewcommand{\hwa@hline@LONE}{\vspace{.25cm} {\hrule height 2pt}
    \vspace{.25cm}}
  \renewcommand{\hwa@headrulewidth}{.7pt}
  \renewcommand{\hwa@hline@LTWO}{\vspace{.5cm} \hrule \vspace{.25cm}}
}{
  \ifthenelse{\equal{\hwa@hlines}{decreased}}{
    \renewcommand{\hwa@hline@LONE}{ \vspace{.25cm} {\hrule height 2pt}
      \vspace{.25cm}}
    \renewcommand{\hwa@headrulewidth}{.7pt}
  }{\ifthenelse{\equal{\hwa@hlines}{header}}{
      \renewcommand{\hwa@headrulewidth}{.7pt}
    }{\ifthenelse{\equal{\hwa@hlines}{none}}{
        \renewcommand{\hwa@headrulewidth}{0pt}
      }{
        \ClassError{homeworkassignment}{Value '\hwa@lines' for key 'hlines'
          is not known}{The option hlines takes an argument to set which
          hlines are drawn. Possible values are 'all','decreased' , 'header', and
          'none'. 'all' is standard.}
      }
    }
    \renewcommand{\hwa@hline@LONE}{~\\\vspace{.5cm}}
  }
  \renewcommand{\hwa@hline@LTWO}{\vspace{.75cm}}
}
%    \end{macrocode}
% \end{macro}
% \label{imp:tikz}
% If tikz is Wanted, load Usefull Styles
%    \begin{macrocode}
\ifhwa@tikz
\RequirePackage{tikz}
\usetikzlibrary{shapes,arrows,positioning,decorations,
  automata,backgrounds,petri,bending,
  shapes.multipart}
\tikzset{
  treenode/.style = {shape=circle, rounded corners,
    draw, align=center},
  graynode/.style = {fill=gray},
  normalnode/.style     = {treenode, font=\Large, bottom color=white},
  array/.style = {rectangle split,
    rectangle split horizontal,
    rectangle split,
    draw}
}
\fi
%    \end{macrocode}
% Make sure that this is the last Package loaded
%    \begin{macrocode}
\RequirePackage{geometry}
\ifhwa@twocolumn
\geometry{top=2cm, bottom=2cm, left=2cm,
    headsep=14pt,hmarginratio={1:1}}
\else
\geometry{top=2cm, bottom=2cm, width=35em,
  headsep=14pt,hmarginratio={4:3}}
\fi
%    \end{macrocode}
% \section{Layout}
% Initially, the homeworkassignment had a verry \emph{special}
% appereance, which became much more customizable with version 3.0, see
% \ref{sec:OPTIONS} if you want to know how.
% \subsection{Headers \& Footers}
% Sets the page-headers.\\
% All headers are cleared before they get any Text --- just to be
% sure.  \\
% 
% The headers have the date on the subject and the author on the right
% side, the tutorial, sheat-title and deadline on the left side, the
% pagenumber is displayed in the right  footer.\\
% If the document is
% twopaged, the informations in the headers are splittet, so that author
% and subject are displayed only on odd pages and the title on even,
% the pagenumber is displayed on the right side on odd pages and on the
% left side on even pages.\\
% On the first page, only the date and tutorial will be displayed in the
% header, the rest of infomration should be in the title.
%    \begin{macrocode}
\fancypagestyle{firstpage}{
  %
  \fancyhf{}
  % clear all six fields
  \renewcommand{\headrulewidth}{\hwa@headrulewidth}
  \renewcommand{\footrulewidth}{0pt}
  \fancyfoot[R]{\thepage}
  \fancyhead[L]{\hwa@tutorium}
  \fancyhead[R]{\@date } }
\fancypagestyle{followingpage}{
  \fancyhf{}
  \ifhwa@twoside % IF
  \fancyhead[RO]{\@author}
  \fancyhead[LO]{\hwa@kurs\\
    \hwa@tutorium}
  \fancyhead[LE]{
    \ifthenelse{\equal{\hwa@sheetTitle}{}}{}{\hwa@sheetTitle\\}
    \GetTranslation{abgabe}: \hwa@abgabe
  }
  \fancyfoot[RO,LE]{\thepage}
  
  \else %ELSE
  
  \fancyhead[R]{\hwa@kurs\\
    \@author}
  \fancyhead[L]{\hwa@tutorium\\
    \ifthenelse{\equal{\hwa@sheetTitle}{}}{}{\hwa@sheetTitle\\}
    \GetTranslation{abgabe}: \hwa@abgabe}
  \fancyfoot[R]{\thepage}
  \fi %ENDIF
  \renewcommand{\headrulewidth}{\hwa@headrulewidth}
  \renewcommand{\footrulewidth}{0pt}
}
\pagestyle{followingpage}
%    \end{macrocode}
% \subsection{Enhance Mathenvironments}
% A couple of thigns, to make math-environments more beautifull and compact.
% \begin{macro}{\theequation}
%   Displays equation-numbers as upper-case roman numbers.
%    \begin{macrocode}
\renewcommand{\theequation}{\Roman{equation}}
%    \end{macrocode}
% \end{macro}
% \begin{macro}{\allowdisplaybreaks}
%   Allow pagebreaks in Mathmode
%    \begin{macrocode}
\allowdisplaybreaks
%    \end{macrocode}
% \end{macro}
% \subsection{fonts}
% \subsection{Serife (Default)}
% \subsubsection{San Serife}
% I fancy the Gillius-Font-Family, so that is the default Sans-Serif
% font, when using XeTeX, The template does default to
% \href{https://www.1001fonts.com/gillius-adf-font.html}{Gillius ADF},
% which is available for free, licensed under the GNU License.
%    \begin{macrocode}
\ifthenelse{\boolean{xetex}}{
  \RequirePackage{fontspec}
  \setsansfont{TeX Gyre Adventor}
  \setmainfont{TeX Gyre Pagella}
  \setmonofont{Fira Mono}
}{ 
  \RequirePackage{tgadventor}
  \RequirePackage{tgpagella}
  \RequirePackage{FiraMono}
}
%    \end{macrocode}
% \subsubsection{Monospace}
%    \begin{macrocode}
%    \end{macrocode}
% \section{Commands}
% \subsection{Constants}
% Defines some constants
% \begin{macro}{\hwa@pointboxsize}
% Explains it self.
%    \begin{macrocode}
\newcommand{\hwa@pointboxsize}{3em}
%    \end{macrocode}
% \end{macro}
% \subsection{Document Informations\label{DOC_INFO_CMDS}}
% \begin{macro}{\subject, \kurs} Sets the subject
%   of the document.  Takes the subject as argument.  Standard
%   Value is
%   \enquote{Kein Kurs}\\
%   |\kurs| is deprecated.\\ \
%    \begin{macrocode}
\newcommand{\hwa@kurs}{?\GetTranslation{subject}?} % To store the value
\newcommand{\subject}[1]{\renewcommand{\hwa@kurs}{#1}}
\newcommand{\kurs}[1]{\subject{#1}}
%    \end{macrocode}
% \end{macro}
% \begin{macro}{\tutorial, \tutorium} Sets the tutorial
% of the author.  Takes it as an argument. Stamdard Value is empty, so
% that this command can be omitted.\\ |\tutorium| is deprecated.\\ \
%    \begin{macrocode}
\newcommand{\hwa@tutorium}{?\GetTranslation{uebungsgruppe}?} % To store the value
\newcommand{\tutorial}[1]{\renewcommand{\hwa@tutorium}{#1}}
\newcommand{\tutorium}[1]{\tutorial{#1}}
%    \end{macrocode}
% \end{macro}
% \begin{macro}{\deadline, \abgabe}
%   Sets the deadline of the document.  Takes it as an argument.
%   Standard value is |\today|.\\ |\abgabe| is deprecated\\
%    \begin{macrocode}
\newcommand{\hwa@abgabe}{\today} % To store the value
\newcommand{\deadline}[1]{\def\hwa@abgabe{#1}}
\newcommand{\abgabe}[1]{\deadline{#1}}
%    \end{macrocode}
% \end{macro}
% 
% \begin{macro}{\sheetTitle}
%   Sets a descriptional Title of the
%   Sheet,
%   will be written in the header of every page.
%    \begin{macrocode}
\newcommand{\hwa@sheetTitle}{}
\newcommand{\sheetTitle}[1]{\def\hwa@sheetTitle{#1}}
%    \end{macrocode}
% \end{macro}
% \subsubsection{Inherited from \texttt{article}}
% \DescribeMacro{\author} Sets the author of the document.\\
% \DescribeMacro{\date} Sets the date of the document.\\
% \subsection{Sectioning\label{SECTIONING_CMDS}}
% Because the class is designed for Assignments, the
% sectioning-commands are different from Article
% \subsubsection{internal
% Sectioning\label{PLAIN_SECTIONING_CMDS}}
% \begin{macro}{\hwa@problem}~\\ \DescribeMacro{\hwa@subproblem}
% \DescribeMacro{\hwa@subsubproblem} These commands work like their
% counterpart in article, except that there will be no number, nor
% will they increase a counter.  Nevertheless, they will be shown
% in the table of contents.  With version 4.0 they were made private,
% because I figured that they are not usefull enough and I am now able
% to introduce environments with the old names\\
%    \begin{macrocode}
\DeclareDocumentCommand\hwa@problem{m o}{\@startsection{problem}%Name
  {1}%Level
  {\z@}%indent
  {-2em \@plus -1em \@minus -1em}%beforeskip
  {1ex \@plus .5ex}%afterskip
  {\normalfont\Large \sffamily\bfseries}%style
  *{#1
    \IfNoValueF{#2}{
      \hfill
     \frame{\framebox[\hwa@pointboxsize]{
         \hfill \normalfont{\large/\small{#2}}}}
    }
  }
  \addcontentsline{toc}{section}{#1}
}

\DeclareDocumentCommand\hwa@subproblem{m o}{\@startsection{subproblem}%Name
  {2}%Level
  {\z@}%indent
  {-1em \@plus -.5em \@minus -.5em}%beforeskip
  {.5ex \@plus .5ex}%afterskip
  {\normalfont\large \sffamily\bfseries}%style
  *{#1
    \IfNoValueF{#2}{
      \hfill \framebox[\hwa@pointboxsize]{
        \hfill\normalfont\large/\small{#2}}
    }
  }
  \addcontentsline{toc}{subsection}{#1}
}

\DeclareDocumentCommand\hwa@subsubproblem{m o}{\@startsection{subsubproblem}%Name
  {3}%Level
  {\z@}%indent
  {-.5em}%beforeskip
  {.5em}%afterskip
  {\normalfont \sffamily\bfseries}%style
  *{#1
    \IfNoValueF{#2}{
      \hfill \framebox[\hwa@pointboxsize]{
        \hfill\normalfont\large/\scriptsize{#2}}
    }
  }
}

%    \end{macrocode}
% \end{macro}
% \begin{macro}{\keyword}
%   Creates a new Paragraph ,which will
%   start with the Argument in Bold, followed by two non-breaking
%   spaces.\\
%    \begin{macrocode}
\newcommand{\keyword}[1]{\@startsection{keyword}%Name
  {4}%Level
  {\parindent}%indent
  {-.1em}%beforeskip
  {\z@}%afterskip
  {\normalfont \sffamily\bfseries}%style
  *{#1~~}
}
%    \end{macrocode}
% 
% The following Macros make use of |\keyword|, so it is suggested to
% use them instead.\\
%
% \DescribeMacro{\solution} \DescribeMacro{\proof}
% \DescribeMacro{\given} \DescribeMacro{\toShow}
% \DescribeMacro{\toDisprove}
% \DescribeMacro{\assumption} \DescribeMacro{\supposeThat} They work like
% |\keyword|, but take only an optional Argument print out
% \enquote{Solution}, \enquote{Proof} \enquote{Given}, \enquote{To
% show}, \enquote{Assumption}, and \enquote{Suppose that},
% respectively \footnote{As of v1.6, Translations are added, depending
% on the choosen Language, there may be an other Text displayed.\\ See
% \autoref{imp:translation} for all Translations}, via |\keyword|. If an
% argument is passed, they print out this argument after the keyword. They
% are not mentioned in the table of contents.
%    \begin{macrocode}
\newcommand{\solution}[1][]{\keyword{\GetTranslation{loesung}\ifstrempty{#1}{}{~#1}:}}
\newcommand{\toShow}[1][]{\keyword{\GetTranslation{zuZeigen}\ifstrempty{#1}{}{~#1}:}}
\newcommand{\toDisprove}[1][]{
  \keyword{\GetTranslation{zuWiderlegen}\ifstrempty{#1}{}{~#1}:}}
\newcommand{\given}[1][]{\keyword{\GetTranslation{gegeben}\ifstrempty{#1}{}{~#1}:}}
\newcommand{\assumption}[1][]{\keyword{\GetTranslation{Annahme}\ifstrempty{#1}{}{~#1}:}}
\newcommand{\supposeThat}[1][]{\keyword{\GetTranslation{Angenommen-dass}\ifstrempty{#1}{}{~#1}}}
%    \end{macrocode}
% \end{macro}
% \subsubsection{\enquote*{better}
% Sectioning\label{BETTER_SECTIONING_CMDS}}
% The following commands are an augmented version of the
% \enquote{plain} commands.\
% \begin{macro}{\newproblem}
%   ~\\
%   \DescribeMacro{\newproblem*} \DescribeMacro{\newsubproblem}
%   \DescribeMacro{\newsubsubproblem} These commands require no argument,
%   and automatically create a numbered title.
%   They have two optional arguments:
%   \texttt{\textbackslash{newproblem}[\#1]\{\#2\}} where \texttt{\#1}
%   is the (sub(sub))problem-number and \texttt{\#2} are the
%   points. If there is a number of Points assigned to a
%   (sub(sub))problem, then the command will generate a box to write
%   the reched number of points down next to it.\\
% 
%   Normally, |\newproblem| adds the new Created Problem to the
%   grading--table (see \autoref{GRADING-TABLE}), |\newproblem*| does not
%   do this.\\
%   
%   They use coutners, of course:\label{imp:Counters}
%    \begin{macrocode}
\newcounter{problem} \setcounter{problem}{0}
\newcounter{subproblem}[problem] \setcounter{subproblem}{0}
\newcounter{subsubproblem}[subproblem] \setcounter{subsubproblem}{0}

%    \end{macrocode}
%    \begin{macrocode}
\DeclareDocumentCommand\newproblem{O{} g}{
  \stepcounter{problem}% to reset the lower counters
  \ifthenelse{\equal{#1}{}}{
    % empty
  }{
    \setcounter{problem}{#1}    
  }
  
  \IfNoValueTF{#2}{
    \hwa@problem{\GetTranslation{aufgabe} \hwa@problemno}
    \addToGradingTable{\# \hwa@problemno}
  }{
    \hwa@problem{\GetTranslation{aufgabe} \hwa@problemno}[#2]
    \addToGradingTable{\# \hwa@problemno}{/#2}
  }
}

\WithSuffix\newcommand\newproblem*[1][]{\stepcounter{problem}
  \ifthenelse{\equal{#1}{}} { } {
    \stepcounter{problem}% to reset the lower counters
    \setcounter{problem}{#1}}
  \hwa@problem{\GetTranslation{aufgabe} \hwa@problemno}
}

\DeclareDocumentCommand\newsubproblem{O{} g}{
  \stepcounter{subproblem}
  \ifthenelse{\equal{#1}{}} { } {
    \setcounter{subproblem}{#1}}
  \IfNoValueTF{#2}{
    \hwa@subproblem{\GetTranslation{aufgabe}
      \hwa@problemno{}.\hwa@subproblemno}
  }
  {
    \hwa@subproblem{\GetTranslation{aufgabe}
      \hwa@problemno{}.\hwa@subproblemno}[#2]
  }
}

\DeclareDocumentCommand\newsubsubproblem{O{} g}{
  \stepcounter{subsubproblem}
  \ifthenelse{\equal{#1}{}} { } {\setcounter{subsubproblem}{#1}}
  \IfNoValueTF{#2}{
    \hwa@subsubproblem{\hwa@subsubproblemno)}
  }
  {
    \hwa@subsubproblem{\hwa@subsubproblemno)}[#2]
  }
}

%    \end{macrocode}
% \end{macro}
% \subsubsection{Even Better Sectioning-Environments}
% \begin{macro}
%   hjk
%    \begin{macrocode}
\NewDocumentEnvironment{problem}{O{} g}{
  \newproblem[#1]{#2}
  \newcommand{\task}[1]{
    \begin{framed}
      \keyword{Problem:} ##1
    \end{framed}
  }
}{}
\NewDocumentEnvironment{problem*}{O{} g}{
  \newproblem*[#1]{#2}
  \newcommand{\task}[1]{
    \begin{framed}
      \keyword{Problem:} ##1
    \end{framed}
  }
}{}
\NewDocumentEnvironment{subproblem}{O{} g}{
  \newsubproblem[#1]{#2}
  \newcommand{\task}[1]{
    \begin{framed}
      \keyword{Problem:} ##1
    \end{framed}
  }
}{}
\NewDocumentEnvironment{subsubproblem}{O{} g}{
  \newsubsubproblem[#1]{#2}
  \newcommand{\task}[1]{
    \begin{framed}
      \keyword{Problem:} ##1
    \end{framed}
  }
}{}
%    \end{macrocode}
% \end{macro}
% 
% \subsection{Useful Macros}
% \subsubsection{\textsc{Quod Erat Demunstrandum}, End of Proof}
% \label{QED}
% \begin{macro}{\QED}~\\ \DescribeMacro{\EOP} \DescribeMacro{\eop}
% Display a flushed-right \textit{QED}, \(\square\), or \(\blacksquare\),
% respectively. |\qed| is not implemented, to keep compatibility to
% several Math-packages, which define the later.
%    \begin{macrocode}
\newcommand{\hwa@QED}{\begin{flushright}
    \textsc{Qed}
  \end{flushright}
}
\newcommand{\QED}{\hwa@QED}

\ifhwa@unicodemath
 \RequirePackage{unicode-math}
\AtBeginDocument{\let\QEDSymbol\QED
  \renewcommand{\QED}{\hwa@QED}
}
\fi

\newcommand{\EOP}{\begin{flushright}
    \(\square\)
  \end{flushright}
}
\newcommand{\eop}{\hfill\(\blacksquare\)}
%    \end{macrocode}
% \end{macro}
% 
% \subsubsection{\textsc{Quod Non Erat Demunstarndum at iucundum est}}
% \begin{macro}{\QNED}~\\ \DescribeMacro{\qned}
%   Display a flushed-right \(triangle\). |\QNED| displays it in a new
%   line, |\qned| at the end of the same line.\\ In Mathematical proofs this
%   symbol is used to mark things, which we did not intend to proof,
%   but are interesting anyway or things wich are not proofed
%   mathematically, but are explained in a ay, whcih lets no doubt on
%   their correctness.
%    \begin{macrocode}
\newcommand{\QNED}{\begin{flushright} \(\triangle\)
  \end{flushright}
}
\newcommand{\qned}{\hfill\(\triangle\)}
%    \end{macrocode}
% \end{macro}
% \subsubsection{Stolen Goods}
% \label{ALLES_NUR_GEKLAUT_EO-EO}
% \begin{center}
% \foreignblockquote{ngerman}{Das ist alles nur geklaut}
% \end{center}
% \begin{flushright}
%   \small{$\sim$Tobias K\"unzel}
% \end{flushright}
% These Commands are not mine, they are all stolen from Alexander
% Bartolomey's\footnote{\enquote{Occloxium} on
% GitHub:\url{https://github.com/occloxium}}
% \texttt{amath}-Class\footnote{\texttt{amath.sty} is part of Alexander
% Bartolomey's Alphabet Classes:
% \url{https://github.com/occloxium/AlphabetClasses}}\\
% \begin{macro}{\N}~\\
%   \newcommand{\N}{\ensuremath{\mathbb{N}}}
%   \newcommand{\Z}{\ensuremath{\mathbb{Z}}}
%   \DescribeMacro{\Z}
%   \newcommand{\R}{\ensuremath{\mathbb{R}}}
%   \DescribeMacro{\R}
%   \newcommand{\Q}{\ensuremath{\mathbb{Q}}}
%   \DescribeMacro{\Q}
%   \newcommand{\Compl}{\ensuremath{\mathbb{C}}}
%   \DescribeMacro{\Compl}
%   \newcommand{\F}{\ensuremath{\mathbb{F}}}
%   \DescribeMacro{\F}
%   \newcommand{\Primes}{\ensuremath{\mathbb{P}}}
%   \DescribeMacro{\Primes}
%   Defines a set of mathematical sets, which are verry usefull (see
%   \autoref{tbl:field-commands})
%   \begin{longtable}[h]{rll}
%     Command & Output & Description\\
%     |\N|&$\N$&Natural Numbers\\
%     |\Z|&$\Z$&Whole Numbers\\
%     |\Q|&$\Q$&Rational Numbers\\
%     |\R|&$\R$&Real Numbers\\
%     |\Compl|&$\Compl$&Complex Numbers\\
%     |\F_n|&$\F_n$&Prime Field to base $n$\\
%     |\Primes|\footnotemark &$\Primes$ & Set of all Primes\\
%     \caption{Field-Commands}
%     \label{tbl:field-commands}
%   \end{longtable}
%   \footnotetext{Has to be
%   $\backslash$\texttt{Primes}, because $\backslash$\texttt{P}
%   is already in use}
%    \begin{macrocode}
\newcommand{\N}{\ensuremath{\mathbb{N}}}
\newcommand{\Z}{\ensuremath{\mathbb{Z}}}
\newcommand{\R}{\ensuremath{\mathbb{R}}}
\newcommand{\Q}{\ensuremath{\mathbb{Q}}}
\newcommand{\Compl}{\ensuremath{\mathbb{C}}}
\newcommand{\F}{\ensuremath{\mathbb{F}}}
% The last one is mine
\newcommand{\Primes}{\ensuremath{\mathbb{P}}}
%    \end{macrocode}
% \end{macro}
% \begin{macro}{\GL}~\\
%   \DescribeMacro{\id}
%   \DescribeMacro{\Var}
%   \DescribeMacro{\Perm}
%   \DescribeMacro{\MComb}
%   \DescribeMacro{\Comb}
%   \DescribeMacro{\Pot}
%   \DescribeMacro{\Map}
%   \DescribeMacro{\Hom}
%   \DescribeMacro{\Ker}
%   \DescribeMacro{\Intpol}
%   \DescribeMacro{\Pol}
%   \DescribeMacro{\Sol}
%   \DescribeMacro{\Bin}
%   \DescribeMacro{\charakteristik}
%   \DescribeMacro{\fo}
%   \DescribeMacro{\first}
%   \DescribeMacro{\la}
%   \newcommand{\diff}[1]{\ensuremath{\frac{d}{d#1}}}
%   \DescribeMacro{\diff}
%   \newcommand{\partdiff}[1]{\ensuremath{\frac{\partial}{\partial#1}}}
%   \DescribeMacro{\partdiff}
%   \newcommand{\dx}{\:dx}
%   \DescribeMacro{\dx}
%   \newcommand{\divides}{\ensuremath{\ |\ }}
%   \DescribeMacro{\divides}
%   \newcommand{\property}{\ensuremath{\ |\ }}
%   \DescribeMacro{\property}
%   \renewcommand{\dim}[1][]{\ensuremath{\text{dim}_{#1}\ }}
%   \DescribeMacro{\dim}
%   \renewcommand{\Im}{\ensuremath{\text{Im}\ }}
%   \DescribeMacro{\Im}
%   \newcommand{\excup}{\ensuremath{\stackrel{.}{\cup}}}
%   \DescribeMacro{\excup}
%   \newcommand{\falls}{\text{\ \GetTranslation{falls}}\ }
%   \DescribeMacro{\falls}
%   Output usefull Plaintext-Operators and Functions.
%   See table \ref{tab:functions}. Require Mathmode\\
%   \begin{longtable}[h]{rl}
%     Command & Output \\
%     |\GL| & $\GL$\\
%     |\id| & $\id$\\
%     |\Var| & $\Var$\\
%     |\Perm|& $\Perm$\\
%     |\Comb|& $\Comb$\\
%     |\MComb|& $\MComb$\\
%     |\Pot|& $\Pot$\\
%     |\Map|& $\Map$\\
%     |\Hom|& $\Hom$\\
%     |\Intpol|& $\Intpol$\\
%     |\Pol|& $\Pol$\\
%     |\Sol|& $\Sol$\\
%     |\Bin| & $\Bin$\\
%     |\charakteristik| & $\charakteristik$\\
%     |\diff{<1>}|&$\diff{<1>}$\\
%     |\partdiff{<1>}|&$\partdiff{<1>}$\\
%     |\divides| and |property| & Prints a vertical line\\
%     |\dx|&$\dx$\\
%     |\excup|&$\excup$\\
%     |\fo|&\(\fo\)\\
%     |\first|&\(\first\)\\
%     |\la|&\(\la\)\\
%     \caption{Common Functions}
%     \label{tab:functions}
%   \end{longtable}
%   |\falls| prints out \foreigntextquote{ngerman}{falls}\footnote{In
%   German, actual Translation may differ}
%    \begin{macrocode}
\DeclareMathOperator{\GL}{GL}
\DeclareMathOperator{\id}{id}
\DeclareMathOperator{\Var}{Var}
\DeclareMathOperator{\Perm}{Perm}
\DeclareMathOperator{\MComb}{MComb}
\DeclareMathOperator{\Comb}{Comb}
\DeclareMathOperator{\Pot}{Pot}
\DeclareMathOperator{\Map}{Map}
\DeclareMathOperator{\Hom}{Hom}
\DeclareMathOperator{\Ker}{Ker}
\DeclareMathOperator{\Intpol}{Intpol}
\DeclareMathOperator{\Pol}{Pol}
\DeclareMathOperator{\Sol}{Sol}
\DeclareMathOperator{\Bin}{Bin}
\DeclareMathOperator{\charakteristik}{char}
\DeclareMathOperator{\fo}{fo}
\DeclareMathOperator{\first}{fi}
\DeclareMathOperator{\la}{la}

\newcommand{\diff}[1]{\ensuremath{\frac{d}{d#1}}}
\newcommand{\partdiff}[1]{\ensuremath{\frac{\partial}{\partial#1}}}
\newcommand{\dx}{\:dx}
\newcommand{\divides}{\ensuremath{\ |\ }}
\newcommand{\property}{\ensuremath{\ |\ }}

\renewcommand{\dim}[1][]{\ensuremath{\text{dim}_{#1}\ }}
\renewcommand{\Im}{\ensuremath{\text{Im}\ }}

\newcommand{\excup}{\ensuremath{\stackrel{.}{\cup}}}
\newcommand{\falls}{\text{\ \GetTranslation{falls}}\ }
%    \end{macrocode}
% \end{macro}
% \subsubsection{Rounding}
% Require Mathmode
% \newcommand{\floor}[1]{\ensuremath{\left\lfloor #1 \right\rfloor}}
% \newcommand{\ceil}[1]{\ensuremath{\left\lceil #1 \right\rceil}}
% \newcommand{\roundHU}[1]{\ensuremath{\left\lceil #1 \right\rfloor}}
% \newcommand{\roundHD}[1]{\ensuremath{\left\lfloor #1 \right\rceil}}
% \begin{longtable}[h]{rll}
%   Command & Output & Meaning\\
%   |\floor{<1>}| & $\floor{<1>}$ & floor |<1>|\\
%   |\ceil{<1>}| & $\ceil{<1>}$ & ceil |<1>|\\
%   |\roundHU{<1>}| & $\roundHU{<1>}$ & Round |<1>| \enquote{half up}
%   ($\floor{<1> + \frac{1}{2}}$)\\
%   |\roundHD{<1>}| & $\roundHD{<1>}$ & Round |<1>| \enquote{half down}
%   ($-\floor{<1> - \frac{1}{2}}$)\\
%   \caption{Rounding Functions}
%   \label{tab:rounding-functions}
% \end{longtable}
%    \begin{macrocode}
\newcommand{\floor}[1]{\ensuremath{\left\lfloor #1 \right\rfloor}}
\newcommand{\ceil}[1]{\ensuremath{\left\lceil #1 \right\rceil}}
\newcommand{\roundHU}[1]{\ensuremath{\left\lceil #1 \right\rfloor}}
\newcommand{\roundHD}[1]{\ensuremath{\left\lfloor #1 \right\rceil}}
%    \end{macrocode}
% \begin{macro}{\bigforall}~\\
%   \DescribeMacro{\bigexists} Redefines big versions of quantors,
%   adds an h-skip to normal version.
%    \begin{macrocode}
\let\oforall\forall
\let\oexists\exists
\renewcommand{\forall}{\ensuremath{\hskip 2pt \oforall \hskip 2pt}}
\renewcommand{\exists}{\ensuremath{\hskip 2pt \oexists \hskip 2pt}}
\newcommand{\bigforall}{\mbox{\raisebox{-2pt}[\height][\depth]{\Large $\mathsurround4pt\forall$}}}
\newcommand{\bigexists}{\mbox{\raisebox{-2pt}[\height][\depth]{\Large $\mathsurround4pt\exists$}}}
%    \end{macrocode}
% \end{macro}
% \subsubsection{ToDos\label{cmd:TODOs}}
% Utility for the Documentation of ToDos
% \begin{macro}{\todo}
% Creates a todo at the location of the command, highlighted in red.
% The ToDos will be listed after maketitle, unless the option
% |todos=nolist| or |todos=none| is specified.
%    \begin{macrocode}
\DeclareDocumentCommand\todo{G{}}{
  \ifthenelse{\boolean{hwa@todos@inplace}}{
    {\color{red}\textbf{~\label{TODO\alph{todoNum}}TODO~}#1~}
    \xdef\hwa@todoList@aux{\hwa@todoList@aux
      \string\item\string\hyperref[TODO\alph{todoNum}]{TODO #1}
    }
    \stepcounter{todoNum}
  }{}
}
%    \end{macrocode}
% Uses the internal |hwa@todo|-counter
%    \begin{macrocode}
\newcounter{todoNum} \setcounter{todoNum}{1}
%    \end{macrocode}
% \end{macro}
% \begin{macro}{\hwa@tableOfTodos}
%   Prints all ToDos
%    \begin{macrocode}
\DeclareDocumentCommand\hwa@tableOfTodos{}{
  \ifthenelse{\boolean{hwa@todos@list}}{
    \ifthenelse{\equal{\hwa@tableOfTodos}{}}{%Nothing
    }{
      {\color{red}
        \hwa@problem{Table of ToDos}
        \begin{itemize}
          \hwa@todoList
        \end{itemize}}
    }
  }{}
}
%    \end{macrocode}
% \end{macro}
% 
% \subsection{Grading Table\label{GRADING-TABLE}}
% This Document-Class is still mainly designed for Homework, so
% it would be nice, if there was a table to write Grades into,
% wouldn't it?\\
%
% \begin{macro}{\addToGradingTable} Adds the given parameter
% as an excercise to the Grading-Table. All Problems, created with
% |\newproblem| are added automatically.
%    \begin{macrocode}
\DeclareDocumentCommand\addToGradingTable{m g}{
  \edef\hwa@gradingtbl@aux@defs{\hwa@gradingtbl@aux@defs|p{\hwa@pointboxsize}}
  \edef\hwa@gradingtbl@aux@lineOne{\hwa@gradingtbl@aux@lineOne{#1} &}
  \IfNoValueTF{#2}{
    \edef\hwa@gradingtbl@aux@lineTwo{\hwa@gradingtbl@aux@lineTwo &}
  }{
    \edef\hwa@gradingtbl@aux@lineTwo{\hwa@gradingtbl@aux@lineTwo\vfill\hfill
      {\string\small #2} &}
  }
}
%    \end{macrocode}
% \end{macro}
% \begin{macro}{\makeGradingTable}
%   Outputs a table to fill in the reached Points.  Only shows Problems
%   created by \texttt{\textbackslash{}newproblem}.  \\Displays the
%   according number of maximum points for each problem, if
%   specified.\\Displays the total number of maximum Problems, if
%   given by Argument Like |\tableofcontent|, it needs a second run of
%   \LaTeX, until all are added.\\
%   It will never overflow the Line-Width, thanks to an |adjustbox|.
%   \texttt{[\#1]}: \emph{Optional.} The total number of points
%   reachable.
%    \begin{macrocode}
\DeclareDocumentCommand\makeGradingTable{o}{
  \begin{table}[hb]
    \centering
    \large
    \begin{adjustbox}{max width=\linewidth}
      \expandafter\tabular\expandafter{\hwa@gradingtbl@defs ||p{\hwa@pointboxsize}|}\hline
      \hwa@gradingtbl@lineOne  \(\Sigma\)       \\\hline\small
      \hwa@gradingtbl@lineTwo  \IfNoValueTF{#1}{~}{\vfill\hfill/#1}\vspace{.15cm}\\\hline
      \endtabular
    \end{adjustbox}
  \end{table}
  }
%    \end{macrocode}
% \end{macro}
% See example documents fot output
% \subsubsection{Internal commands}
% \begin{macro}{\hwa@gradingtbl@...}
% Defines macros whose contents will be written to the AUX-File and
% read in the next run, and the usable commands. The later will
% contain the information, but have to be defined (incase the aux-file
% does not exist)
%    \begin{macrocode}
\edef\hwa@gradingtbl@aux@defs{}
\newcommand{\hwa@gradingtbl@aux@lineOne}{}
\newcommand{\hwa@gradingtbl@aux@lineTwo}{}

\edef\hwa@gradingtbl@defs{}
\newcommand{\hwa@gradingtbl@lineOne}{}
\newcommand{\hwa@gradingtbl@lineTwo}{}
%    \end{macrocode}
% \end{macro}
% \begin{macro}{\hwa@todoList@...}
% See |\hwa@gradingtlb@...|.
%    \begin{macrocode}
\newcommand{\hwa@todoList}{}
\newcommand\hwa@todoList@aux{}
% \end{macro}
% \begin{macro}{\write\@auxout}
%   Write to aux
%    \begin{macrocode}
\AtEndDocument{%
  \immediate\write\@auxout{%
    \gdef\string\hwa@gradingtbl@defs{\hwa@gradingtbl@aux@defs}
  }
  \immediate\write\@auxout{%
    \gdef\string\hwa@gradingtbl@lineOne{\hwa@gradingtbl@aux@lineOne}%
  }
  \immediate\write\@auxout{%
    \gdef\string\hwa@gradingtbl@lineTwo{\hwa@gradingtbl@aux@lineTwo}%
  }
  \immediate\write\@auxout{%
    \gdef\string\hwa@todoList{\hwa@todoList@aux}%
  }
}
%    \end{macrocode}
% \end{macro}
% \subsection{Title}
% \begin{macro}{\maketitle}
%   Overrides maketitle.
%    \begin{macrocode}
\renewcommand{\maketitle} {
  \thispagestyle{firstpage}
  \ifhwa@twocolumn{
    \twocolumn[{
      \hwa@maketitletext
    }]
  }\else{
    \hwa@maketitletext
  }\fi
  \hwa@tableOfTodos
}
%    \end{macrocode}
% \end{macro}
% \begin{macro}{\hwa@maketitletext}
%   Prints out the title with author etc. Used to reduce code
%   duplication for two- and onecolumn styles
%    \begin{macrocode}
\newcommand{\hwa@maketitletext}{
  \begin{centering}
    \huge{\textsf{\textbf{\hwa@kurs}}}\hwa@hline@LONE \large
    \ifthenelse{\equal{\hwa@sheetTitle}{}}{}{\textbf{\hwa@sheetTitle}\\}
    \GetTranslation{abgabe}: \hwa@abgabe\\
    \hwa@hline@LTWO
    \normalsize{\@author}\\
    \hwa@hline@LTWO \normalsize
  \end{centering}
}
\ifthenelse{\boolean{hwa@punchmark}}{
  \newcommand{\hwa@punchmarkRad}{3mm}
  \newcommand{\hwa@punchmarkDistanceX}{12mm}
  \newcommand{\hwa@punchmarkDistanceY}{40mm}
  \AtBeginDocument{
  % Where will the punch be?
  \AddToShipoutPictureBG*{\AtPageUpperLeft{
      \put(\LenToUnit{\hwa@punchmarkDistanceX-\hwa@punchmarkRad*2},\LenToUnit{-.5\paperheight-\hwa@punchmarkDistanceY-\hwa@punchmarkRad}){\tikz{\draw (0,0) circle (\hwa@punchmarkRad);}}
      \put(\LenToUnit{\hwa@punchmarkDistanceX-\hwa@punchmarkRad*2},\LenToUnit{-.5\paperheight+\hwa@punchmarkDistanceY-\hwa@punchmarkRad}){\tikz{\draw (0,0) circle (\hwa@punchmarkRad);}}}}
  % Punch-Positioningmark
  \AddToShipoutPictureBG*{\AtPageUpperLeft{
      \put(\LenToUnit{5mm},\LenToUnit{-.5\paperheight}){\tikz{\draw (0,0) -- (5mm,0);}}}}
  }
}{
}
%    \end{macrocode}
% \end{macro}
% \subsection{Counters}
% The actual counters are defined in \autoref{imp:Counters}.
% \begin{macro}{Counter--Commands}
%   These are used to output the Exercise numbers in the desired style
%    \begin{macrocode}
\newcommand{\hwa@problemno}{\arabic{problem}}
\newcommand{\hwa@subproblemno}{\alph{subproblem}}
\newcommand{\hwa@subsubproblemno}{\roman{subsubproblem}}
%    \end{macrocode}
%
% \begin{macro}{\hwa@parseCounterStyle}
%   This takes a style-input (\#1), one of the three previous defined
%   commands (\#2) and the coresponding counter (\#3) to redefine \#1,
%   so that it corresponds to \#2.  See
%   \ref{RE-DEF-COUNTER-CMDS-IMPLM} for example usement.
%
%    \begin{macrocode}
\newcommand{\hwa@parseCounterStyle}[3]{
  \ifthenelse{\equal{#1}{arabic}}{ \renewcommand{#2}{\arabic{#3}} }{
    \ifthenelse{\equal{#1}{roman}}{ \renewcommand{#2}{\roman{#3}} }{
      \ifthenelse{\equal{#1}{alph}}{ \renewcommand{#2}{\alph{#3}} }{
        \ifthenelse{\equal{#1}{Alph}}{ \renewcommand{#2}{\Alph{#3}} }{
          \ifthenelse{\equal{#1}{Roman}}{
            \renewcommand{#2}{\Roman{#3}} }{
            \ClassError{homeworkassignment}{Invalid Value #1 for
              option Counter-Styling}{Possible Values are alph,
              arabic, Arabic, roman or Roman.}  } } } } } }
%    \end{macrocode}
% \end{macro}
%   Redefines the three counter-commands:
%    \begin{macrocode}
\hwa@parseCounterStyle{\hwa@problemsty}{\hwa@problemno}{problem}
\hwa@parseCounterStyle{\hwa@subproblemsty}{\hwa@subproblemno}{subproblem}
\hwa@parseCounterStyle{\hwa@subsubproblemsty}{\hwa@subsubproblemno}{subsubproblem}
%    \end{macrocode}
% \end{macro}
% \section{Environments}
% \subsection{Proof}
% Used for proofes. Starts bth proof and ends with a End-Of-Proof
% symbol.
%    \begin{macrocode} 
\NewDocumentEnvironment{proof}{G{\GetTranslation{beweis}} O{\QED}}
{
  \keyword{#1:~~}
}
{
  #2
}
%    \end{macrocode}
% \subsection{Proof by contradiction}
% Used for proofes. Starts bth proof and ends with a End-Of-Proof
% symbol.
%    \begin{macrocode} 
\NewDocumentEnvironment{contradiction}{}
{
  \begin{proof}{\GetTranslation{beweis}~\GetTranslation{per}~\GetTranslation{Widerspruch}}[\hfill\lightning\\]
  }
  {
  \end{proof}
}
%    \end{macrocode}
% \pagebreak
% \section{Development and support}
%
% The package is developed at \emph{GitHub}:
% \begin{quote}
%   \url{https://github.com/ACHinrichs/LaTeX-templates}
% \end{quote}
% Please refer to that site for any bug report or development
% information.
%
% \section{Changelog}
% \begin{description}
% \item[v1.0 - 2016/10/23] Initial
% \item[v1.1 - 2016/11/02] ...
% \item[v1.2 - 2016/11/03] ...
% \item[v1.3 - 2016/12/01]
%   \begin{itemize}
%   \item Provide the Class as .dtx file and more
%   \end{itemize}
% \item[v1.4 - 2017/04/29]
%   \begin{itemize}
%   \item \enquote{Minor} bugfixes
%   \end{itemize}
% \item[v1.5 - 2017/04/29]
%   \begin{itemize}
%   \item Problems are displayed in the table of contents. Type of
%     numeration is now configurable.
%   \end{itemize}
% \item[v1.5.1 - 2017/04/29]
%   \begin{itemize}
%   \item Bugfix
%   \end{itemize}
% \item[v1.5.2 - 2017/04/29]
%   \begin{itemize}
%   \item Add version-number
%   \end{itemize}
% \item[v1.6 - 2017/05/02]
%   \begin{itemize}
%   \item Add Translations (German and English)
%   \item Add |\given| and |\toShow|
%   \item Add |\QED|, |\EOP|, and |\eop|
%   \end{itemize}
% \item[v1.6.3 - 2017/05/05]
%   \begin{itemize}
%   \item Bugfixes
%   \end{itemize}
% \item[v1.6.4 - 2017/05/09]
%   \begin{itemize}
%   \item Change |\eop| to be in the same line
%   \end{itemize}
% \item[v1.7 - 2017/05/09]
%   \begin{itemize}
%   \item Add |\QNED|
%   \end{itemize}
% \item[v2.0 - 2017/05/23]\enquote{Layout 2.0}
%   \begin{itemize}
%   \item Change Margins
%   \item Add Option to select older Page-Style
%   \item Change standardlayout to twocolumn and twoside
%   \item \st{Steal} Use Macros by Alexander Bartolomey (See \ref{ALLES_NUR_GEKLAUT_EO-EO})
%   \item Add some TikZ-Styles
%   \item Add round functions
%   \end{itemize}
% \item[v2.2 - 2017/06/17]
%   \begin{itemize}
%   \item Add Grading-table
%   \item Add |\keyword|, |\assumption|, and |\supposeThat|
%   \item Add |\newproblem*|
%   \item Add |\sheetTitle|
%   \item Change equation-numbering to uppercase roman
%   \end{itemize}
% \item[v2.2.1 - 2017/06/20]
%   \begin{itemize}
%   \item Fix error with commands like |\solution| and |\keyword|.
%   \end{itemize}
% \item[v2.4 - 2017/04/07]
%   \begin{itemize}
%   \item Fix math alignment
%   \item Add option for flushed left equations
%   \item Update amath port to use
%   \end{itemize}
% \item[v3.0 - 2017/12/26] \enquote{WS 2017}
%   \begin{itemize}
%   \item Rename to \texttt{homeworkassignment}
%   \item Add Environment for various proofs
%   \item Add points for exercises and a place to fill them in
%   \item Add option to remove or decrease or remove the hlines
%   \item Remove legacy styles
%   \item Rework the documentation
%   \item Beautify Maths
%   \item Fix OneColumn-Maktitle-Bug
%   \item Fix Subproblem-Counter not beeing reset
%   \item Merry Christmas!
%   \end{itemize}
% \item[v3.2 - 2018/12/06] Nikolaus Release
%   \begin{itemize}
%   \item Make XeLaTex-Compatible
%     \begin{itemize}
%     \item Rename |\C| to |\Compl|, because of a |unicode-math|
%     incompatibility
%     \end{itemize}
%   \item Fix |\newproblem| requiring a Problem-Number
%   \item Add |\toDisprove| macro similar to the |\toShow| macro
%   \item Add option for punchmarks
%   \item Add option to load unicode-math and work around a incompability
%   \end{itemize}
% 
% \end{description}
% \subsection{Version--Scheme}Since Version 2.0 the following version--scheme
% applies:
% \paragraph{Major Version} has to be increased, if
% \begin{itemize}
% \item there are changes, which create visible changes in the output
%   of existing documents (except for bugfixes), or
% \item a command is removed or changed in a way, that existing
%   documents do not compile with the new version.
% \end{itemize}
% \paragraph{Minor Version} has to be increased, if
% \begin{itemize}
% \item new backwards compatible commands are introduced
%   \begin{itemize}
%   \item Bugfixes may be introduced too.
%   \end{itemize}
% \end{itemize}
% The minor version of stable releases is always even, the minor
% version of developtment versions is always odd. (0 counts as even). 
% \paragraph{Patches} May be introduced on Stable Branch. With every
% non-document-breaking bugfix, the patch--number has to be
% incremented.\\
% Because Fixing Bugs is a part of developtment, developtment-versions
% do not have numeric patch--numbers, but alphabetic identifiers,
% directly after the minor--version.
% \pagebreak
% \section{Translations\label{imp:translation}}
% Homeworkassignment currently supports English and German, fallback
% language is German. Unfortunatly these two are the only Languages I
% am capable of translating reliable, so if you want to use an other
% language, I would be verry happy if you would help me to translate
% homeworkassignment! Please open an issue, author a pull-request or
% send me an e-mail.
%    \begin{macrocode}
\DeclareTranslationFallback{aufgabe}{Aufgabe}
\DeclareTranslationFallback{loesung}{L\"osung}
\DeclareTranslationFallback{beweis}{Beweis}
\DeclareTranslationFallback{uebungsgruppe}{\"Ubungsgruppe}
\DeclareTranslationFallback{abgabe}{Abgabe}
\DeclareTranslationFallback{zuZeigen}{Zu zeigen}
\DeclareTranslationFallback{zuWiderlegen}{Zu widerlegen}
\DeclareTranslationFallback{gegeben}{Gegeben}
\DeclareTranslationFallback{falls}{falls}
\DeclareTranslationFallback{Annahme}{Annahme}
\DeclareTranslationFallback{Angenommen-dass}{Anngenommen, dass}
\DeclareTranslationFallback{per}{per}
\DeclareTranslationFallback{Widerspruch}{Widerspruch}

\DeclareTranslation{German}{aufgabe}{Aufgabe}
\DeclareTranslation{German}{loesung}{L\"osung}
\DeclareTranslation{German}{beweis}{Beweis}
\DeclareTranslation{German}{uebungsgruppe}{\"Ubungsgruppe}
\DeclareTranslation{German}{abgabe}{Abgabe}
\DeclareTranslation{German}{zuZeigen}{Zu zeigen}
\DeclareTranslation{German}{zuWiderlegen}{Zu widerlegen}
\DeclareTranslation{German}{gegeben}{Gegeben}
\DeclareTranslation{German}{falls}{falls}
\DeclareTranslation{German}{Falls}{Falls}
\DeclareTranslation{German}{Annahme}{Annahme}
\DeclareTranslation{German}{Angenommen-dass}{Anngenommen, dass}
\DeclareTranslation{German}{per}{per}
\DeclareTranslation{German}{Widerspruch}{Widerspruch}

\DeclareTranslation{English}{aufgabe}{Problem}
\DeclareTranslation{English}{loesung}{Solution}
\DeclareTranslation{English}{beweis}{Proof}
\DeclareTranslation{English}{uebungsgruppe}{Tutorial}
\DeclareTranslation{English}{abgabe}{Deadline}
\DeclareTranslation{English}{zuZeigen}{To show}
\DeclareTranslation{English}{zuWiderlegen}{To disprove}
\DeclareTranslation{English}{gegeben}{Given}
\DeclareTranslation{English}{falls}{if}
\DeclareTranslation{English}{Falls}{If}
\DeclareTranslation{English}{Annahme}{Assumption}
\DeclareTranslation{English}{Angenommen-dass}{Suppose that}
\DeclareTranslation{English}{per}{by}
\DeclareTranslation{English}{Widerspruch}{contradiction}
%    \end{macrocode}
%\section*{End}
% \textit{The End}
%    \begin{macrocode}
\endinput
%    \end{macrocode}
