% This file provides an example Beamer presentation using the RWTH theme
% showcasing some of the more common options, similar to the Powerpoint version
% 12.11.2014: Revision 1 (Harold Bruintjes, Tim Lange)

% For RWTH, beamer should be loaded with class option t (top)
\documentclass[t]{beamer}
%\usepackage[utf8]{inputenc}
\RequirePackage{ifthen}
% Use fontspec to get Arial font
% Requires use of XeLaTeX
\usepackage{fontspec}
\setmainfont{Product Sans}
\setsansfont{Product Sans}
% Also force Arial for math for a more consistent look
%\usepackage{unicode-math}
%\setmathfont{Arial}

% German style date formatting (footer)
\usepackage[ddmmyyyy]{datetime}
\renewcommand{\dateseparator}{.}

% Format the captions used for figures etc.
\usepackage[compatibility=false]{caption}
\captionsetup{singlelinecheck=off,justification=raggedleft,labelformat=empty,labelsep=none}


% Load the actual RWTH theme. Suggested is to load the full theme,
% as it requires some specific dimensions
\usetheme{projektion}

% Setup presentation information
\title{Testing: Gitlab-CI}
\subtitle{}
\date[RWTH]{Maschinelle Übersetzung\enskip \enskip\today}
\author{Alexander Bartolomey}
\institute[RWTH]{RWTH Aachen University}

% Set the logo to the file `logo`
% It will be scaled automatically
\logo{}

% Uncomment this if you want a TOC at every section start
%\AtBeginSection{\frame{
%    \frametitle{Content}
%    \tableofcontents[currentsection]
%}}

\begin{document}

% Note: Title pages should be created as plain
% Title page with a blue bar
\setbeamercolor{title page bar}{fg=polynom-blue}
\setbeamertemplate{title page}[projektion]{}
\begin{frame}[plain]
\titlepage{}
\end{frame}

% Title page with a 1/3rd size picture
%\setbeamercolor{title page bar}{fg=white}
%\setbeamertemplate{title page}[rwth][title_small]{}
%\begin{frame}[plain]
%\titlepage
%\end{frame}

% Title page with a 2/3rd size picture
%\setbeamertemplate{title page}[rwth2][title_large]{}
%\begin{frame}[plain]
%\titlepage
%\end{frame}

% Start a new section (text is displayed on top of a frame)
% Frame with items
\begin{frame}
  \framesubtitle{Test:}
  \frametitle{Inhalt}
\end{frame}

\section{Pytest}

\begin{frame}
  \frametitle{Pytest}
  Auf dieser Folie steht ein Fließtext, auch wenn man dies eigentlich
  bei Folien lassen sollte.  Da hier aber ein Fließtext sthet, kann
  ich die rechten und linken ränder so wie das umbruchverhalten begutachten.
\end{frame}

% \begin{frame}{Warum Testing?}
%   \begin{itemize}
%     \item \emph{Spontante Desintegration} der Ariane 5
%     \item \emph{Spontante Desintegration} des Mars Climate Orbiters
%     \item \emph{Spontante Desintegration} der Delta 3
%   \end{itemize}
% \end{frame}

\begin{frame}{Frame title}
  \begin{itemize}
    \item First\alert{level}
    \begin{itemize}
      \item Second level
      \begin{itemize}
      \item Third level
    \end{itemize}
  \end{itemize}
\end{itemize}
\end{frame}

\setbeamertemplate{final page}[projektion][Noch Fragen?]{Danke für
  eure Aufmerksamkeit!}
\begin{frame}[plain]
  \usebeamertemplate{final page}
\end{frame}

\end{document}
