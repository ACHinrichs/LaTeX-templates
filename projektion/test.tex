% This file provides an example Beamer presentation using the Polynom Theme
% showcasing some of the more common options, similar to the Powerpoint version
% 

% For Polynom, beamer should be loaded with class option t (top)
\documentclass[t]{beamer}
%\usepackage[utf8]{inputenc}
\RequirePackage{ifthen}

\usepackage{lmodern}
% Use fontspec to get Arial font
% Requires use of XeLaTeX
\usepackage{fontspec}
%\setmainfont{Product Sans}
\setsansfont[]{Product Sans}
\setmonofont[]{Fira Mono}
% Also force Arial for math for a more consistent look
\usefonttheme[onlymath]{serif}
\boldmath{}
% German style date formatting (footer)
\usepackage[ddmmyyyy]{datetime}
\renewcommand{\dateseparator}{.}

% Format the captions used for figures etc.
\usepackage[compatibility=false]{caption}
\captionsetup{singlelinecheck=off,justification=raggedleft,labelformat=empty,labelsep=none}

% Load the actual Polynom theme. Suggested is to load the full theme,
% as it requires some specific dimensions
\usetheme{projektion}

% Setup presentation information
\title{Testing: Gitlab-CI}
\subtitle{Maschinelle\\Übersetzung}
\date[Polynom]{\today}
\author{Alexander Bartolomey}
\institute[Polynom]{Polynom Aachen University}

\begin{document}

% Note: Title pages should be created as plain
% Title page with a blue bar
\setbeamercolor{title page bar}{fg=polynom-blue}
\setbeamertemplate{title page}[projektion]{}

\begin{frame}[plain]
  \titlepage{}
\end{frame}

\begin{frame}{Gliederung}
  \tableofcontents[currentsubsubsection]
\end{frame}

\section{Pytest}

\begin{frame}
  \framesubtitle{Test:}
  \frametitle{Inhalt}
  Damit die gestapelte Überschrift benutzt wird, muss \begin{itemize}
    \item die obere Zeile mit \texttt{\textbackslash{}framesubtitle\{...\}} und die
    \item untere Hauptzeile mit \texttt{\textbackslash{}frametitle\{...\}} angegeben werden.
  \end{itemize} 
\end{frame}

\begin{frame}
  \frametitle{Pytest}
  Auf dieser Folie steht ein Fließtext, auch wenn man dies eigentlich
  bei Folien lassen sollte.  Da hier aber ein Fließtext steht, kann
  ich die rechten und linken Ränder so wie das Umbruchverhalten begutachten. 
  Lorem ipsum dolor sit amet, consetetur sadipscing elitr, sed diam nonumy eirmod 
  tempor invidunt ut labore et dolore magna aliquyam erat, sed diam voluptua. 
  At vero eos et accusam et justo duo dolores et ea rebum. Stet clita kasd gubergren, 
  no sea takimata sanctus est Lorem ipsum dolor sit amet. Lorem ipsum dolor sit amet, 
  consetetur sadipscing elitr, sed diam nonumy eirmod tempor invidunt ut labore 
  et dolore magna
\end{frame}

\begin{frame}
  \frametitle{Mathe in \texttt{Polynom}}
  Mathe sieht super aus dank Serifen und fett gedruckten Lettern, hier ist ein Beispiel:
  \[A = U\Sigma V^T\]
  für jede Matrix \(A \in \mathbb{R}^{m\times n}\) mit \[\Sigma:=\mathrm{diag}(\sigma_1,\dots,\sigma_p) \in \mathbb{R}^{m\times n}, p = \min(m,n)\]
\end{frame}

\subsection{Testing}

\begin{frame}
  \frametitle{Warum Testing?}
  \begin{itemize}
    \item \textbf{Spontante Desintegration} der Ariane 5
    \item \textbf{Spontante Desintegration} des Mars Climate Orbiters
    \item \textbf{Spontante Desintegration} der Delta 3
  \end{itemize}
\end{frame}

\begin{frame}
  \frametitle{Listen- \& Einrückungsdemo}
  \begin{itemize}
    \item First level
    \begin{itemize}
      \item Second level
      \begin{itemize}
        \item Third level
      \end{itemize}
    \end{itemize}
  \end{itemize}
\end{frame}

\setbeamertemplate{banner page}[projektion][Fragen?]
\begin{frame}
  \usebeamertemplate{banner page}
\end{frame}

\setbeamertemplate{banner page invert}[projektion][Antworten!]
\begin{frame}
  \usebeamertemplate{banner page invert}
\end{frame}

\end{document}
