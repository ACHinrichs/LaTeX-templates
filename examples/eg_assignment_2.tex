\documentclass[a4paper,twoside]{homeworkassignment}
\usepackage[ngerman]{babel}
\usepackage{tikz}
\usetikzlibrary{%
  arrows,%
  shapes.misc,% wg. rounded rectangle
  shapes.arrows,%
  chains,%
  matrix,%
  positioning,% wg. " of "
  scopes,%
  decorations.pathmorphing,% /pgf/decoration/random steps | erste Graphik
  shadows%
}
\usepackage{amsmath}
\usepackage[autostyle,german = guillemets]{csquotes}

\author{Autor Eins 1701\\ Autor Zwei 74656}
\date{\today}
\abgabe{\today}
\tutorium{\"Ubungsgruppe 42}
\kurs{Programmierung}
\begin{document}


\tikzset{
  nonterminal/.style={
    % The shape:
    rectangle,
    % The size:
    minimum size=6mm,
    % The border:
    very thick,
    draw=red!50!black!50,         % 50% red and 50% black,
                                  % and that mixed with 50% white
    % The filling:
    top color=white,              % a shading that is white at the top...
    bottom color=red!50!black!20, % and something else at the bottom
    % Font
    font=\itshape
  },
  terminal/.style={
    % The shape:
    rounded rectangle,
    minimum size=6mm,
    % The rest
    very thick,draw=black!50,
    top color=white,bottom color=black!20,
    font=\ttfamily},
  skip loop/.style={to path={-- ++(0,#1) -| (\tikztotarget)}}
}

{
  \tikzset{terminal/.append style={text height=1.5ex,text depth=.25ex}}
  \tikzset{nonterminal/.append style={text height=1.5ex,text depth=.25ex}}
}


\maketitle
\newproblem
\newsubproblem
\newsubsubproblem
\begin{align*}
  &&S_2 \\
  S_2 \rightarrow & A.S_2 & A.S_2\\
  A   \rightarrow & B & B.S_2\\
  B   \rightarrow & p & p.S_2\\
  S_2 \rightarrow & A.S_2 & p.A.S_2\\
  A   \rightarrow & B & p.B.S_2\\
  B   \rightarrow & q & p.q.S_2\\
  S_2 \rightarrow & A. & p.q.A.\\
  A   \rightarrow & B:-B & p.q.B:-B.\\
  B   \rightarrow & r & p.q.r:-B.\\
  B   \rightarrow & q & p.q.r:-q.\\  
\end{align*}
Der Ausdruck wird akzeptiert.  

\begin{align*}
\mathcal{W}(p.q.r:-q) = & \mathcal{W}(p.q.)\cup\{r\}\\
= & \mathcal{W}(p.)\cup{q}\cup\{r\}\\
= & \{p\}\cup\{q\}\cup\{r\}\\
= & \{p,q,r\}\\
\end{align*}

\newsubsubproblem

\begin{align*}
         &&S_2\\
S_2 \rightarrow & A.S_2 & A.S_2\\ 
A   \rightarrow & B:-B & B:-B.S_2\\
B   \rightarrow & q & q:-B-S_2\\
B   \rightarrow & p & q:-p.S_2\\
S_2 \rightarrow & A. & q:-p.A.\\
A   \rightarrow & B:-B & q:-p.B:-B.\\
B   \rightarrow & p & q:-p.p:-B.\\
B   \rightarrow & q & q:-p.p:-q.\\
\end{align*}
Der Ausdruck wird akzeptiert.

\begin{align*}
\mathcal{W}(q:-p.p:-q.) & = \mathcal{W}(q:-p.)\\
& = \emptyset 
\end{align*}

\newsubsubproblem

\begin{align*}
&&S_2\\
S_2 \rightarrow & A.S_2 & A.S_2\\
A   \rightarrow & B:-B & B:-B.S_2\\
B   \rightarrow & q & q:-B.S_2\\
B   \rightarrow & p & q:-p.S_2\\
S_2 \rightarrow & A. & q:-p.A.\\
A   \rightarrow & B & q:-p.B.\\
B   \rightarrow & p & q:-p.p.\\
\end{align*}
Der Ausdruck wird Akzeptiert.

\begin{align*}
\mathcal{W}(q:-p.p.) & = \mathcal{W}(q:-p.) \cup \{ p \}\\
& = \emptyset \cup \{ p \}\\
& = \{ p \}\\
\end{align*}

\newsubsubproblem

Der Ausdruck wird nicht Akzeptiert, da \enquote{t} kein Symbol des Alphabetes ist.
\newsubproblem

Sei $\mathcal{S}$ eine Sprache und $\mathcal{P}$ ein Programm.

Zu zeigen:\\

\begin{align*}
&&\mathcal{P} \text{ ist semantisch korrekt bzgl. } \mathcal{S} \Rightarrow & \mathcal{P} \text{ ist syntaktisch korrekt}\\
\Leftrightarrow&&\mathcal{P} \text{ ist syntaktisch Falsch} \Rightarrow & \mathcal{P} \text{ ist semantisch falsch} & \text{(entspricht Def.)}\\
&&&&qed
\end{align*}
\subsection*{c)}
Seien $\mathcal{A}_1$ und $\mathcal{A}_2$ zwei Ausdrücke in einer Sprache und es gelte:
\begin{align*}
&&  \mathcal{W}(\mathcal{A}_1) \neq \mathcal{W}(\mathcal{A}_2) & \Rightarrow \mathcal{A}_1 \neq \mathcal{A}_2\\
\text{dann gilt auch: }&& \mathcal{A}_1 = \mathcal{A}_2 &  \Rightarrow \mathcal{W}(\mathcal{A}_1) = \mathcal{W}(\mathcal{A}_2) \\
&&&&qed 
\end{align*}

\newproblem[3] % Ueberspringt aufgabe 2

\newsubproblem
$G = (\{S,A,B\},\{a,b\},P,S\}$ mit den Produktionsregeln $P$:
\begin{align*}
  S \rightarrow & A \\
  S \rightarrow & B \\
  A \rightarrow & aAb\\
  A \rightarrow & AA\\
  A \rightarrow & a\\
  B \rightarrow & \varepsilon\\
  B \rightarrow & Bb\\
\end{align*}

\newsubproblem

\begin{align*}
S_1& = ( \{ b \} | S_2)\\
S_2& = [ [S_2] a [S_2] b [S_2] ]\\  
\end{align*}

\newpage
\newsubproblem

\begin{figure}[h]

\begin{tikzpicture}[
    >=latex,thick,
    /pgf/every decoration/.style={/tikz/sharp corners},
    fuzzy/.style={decorate,
        decoration={random steps,segment length=0.5mm,amplitude=0.15pt}},
    minimum size=6mm,line join=round,line cap=round,
    terminal/.style={rectangle,draw,fill=white,fuzzy,rounded corners=3mm},
    nonterminal/.style={rectangle,draw,fill=white,fuzzy},,
    node distance=4mm,
  ]

    \ttfamily
    \begin{scope}[start chain,
            every node/.style={on chain},
            terminal/.append style={join=by {->,shorten >=-1pt,
                fuzzy,decoration={post length=4pt}}},
            nonterminal/.append style={join=by {->,shorten >=-1pt,
                fuzzy,decoration={post length=4pt}}},
            support/.style={coordinate,join=by fuzzy}
        ]

        \node [support]             (start)        {};
        \node [support,xshift=5mm]  (after start2)   {};
        \node [support,xshift=5mm]  (after start)   {};
        \node [terminal,xshift=5mm]  (b)      {b};
        \node [support,xshift=5mm]  (before end)   {};
        \node [support,xshift=5mm]  (before end2)   {};
        \node [coordinate,join=by ->] (end)        {};
    \end{scope}
    \node (s2)  [nonterminal,above=of b] {$S_2$};
    \node (support) [below=of b] {};

    \begin{scope}[->,decoration={post length=4pt},rounded corners=2mm,
            every path/.style=fuzzy]
        \draw (after start2)    |- (s2);
        \draw (s2) -| (before end2);
        \draw (before end)  -- +(0,-.7) -| (after start);
    \end{scope}
\end{tikzpicture}
\caption{Regel $S_1$}
\end{figure}

\begin{figure}[h]

\begin{tikzpicture}[
    >=latex,thick,
    /pgf/every decoration/.style={/tikz/sharp corners},
    fuzzy/.style={decorate,
        decoration={random steps,segment length=0.5mm,amplitude=0.15pt}},
    minimum size=6mm,line join=round,line cap=round,
    terminal/.style={rectangle,draw,fill=white,fuzzy,rounded corners=3mm},
    nonterminal/.style={rectangle,draw,fill=white,fuzzy},,
    node distance=4mm,
  ]

    \ttfamily
    \begin{scope}[start chain,
            every node/.style={on chain},
            terminal/.append style={join=by {->,shorten >=-1pt,
                fuzzy,decoration={post length=4pt}}},
            nonterminal/.append style={join=by {->,shorten >=-1pt,
                fuzzy,decoration={post length=4pt}}},
            support/.style={coordinate,join=by fuzzy}
        ]

        \node [support]             (start)        {};
        \node [support,xshift=5mm]  (after start)  {};
        \node [support,xshift=5mm]  (line S2_1)    {};
        \node [support,xshift=5mm]  (before a)    {};
        \node [terminal,xshift=5mm] (a)            {a};
        \node [support,xshift=5mm]  (after a)      {};
        \node [support,xshift=5mm]  (line S2_2)    {};
        \node [support,xshift=5mm]  (before b)    {};
        \node [terminal,xshift=5mm] (b)            {b};
        \node [support,xshift=5mm]  (before end)   {};
        \node [coordinate,join=by ->] (end)        {};
    \end{scope}
    \node (s2_1)  [nonterminal,above=of line S2_1] {$S_2$};
    \node (s2_2)  [nonterminal,above=of line S2_2] {$S_2$};

    \begin{scope}[->,decoration={post length=4pt},rounded corners=2mm,
            every path/.style=fuzzy]
        \draw (after start)    |- (s2_1);
        \draw (s2_1) -| (before a);
        \draw (after a)    |- (s2_2);
        \draw (s2_2) -| (before b);
        \draw (before end)  -- +(0,-.7) -| (after start);
    \end{scope}
\end{tikzpicture}
\caption{Regel $S_2$}
\end{figure}
\end{document}
