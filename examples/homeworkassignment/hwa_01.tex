\documentclass[listings, a4paper, 11pt]{homeworkassignment}
%\usepackage[utf8]{inputenc}
\usepackage[ngerman]{babel}
% \usepackage[OT1]{fontenc}
% \usepackage{lmodern}
\usepackage{framed}
\usepackage{csquotes}
\usepackage{ stmaryrd }
\usepackage{float}
\usepackage{pdflscape}
\usepackage{rotating}
\usepackage[normalem]{ulem}

\usepackage{tikz}
\usetikzlibrary{arrows,automata,shapes,snakes}

\date{\today}
\author{
  Georg Dorndorf\\
  Adrian Hinrichs}
\subject{Berechenbarkeit und Komplexität}
\tutorial{Übungsgruppe 1}

\newcommand{\task}[1]{
  \begin{framed}
    \keyword{Problem:} #1
  \end{framed}
}
\newcommand*\bitneg[1]{\overline{#1}}
\newcommand*\card[1]{\left|#1\right|}
\newcommand{\norm}[1]{\left\lVert#1\right\rVert}
\sheetTitle{\"Ubung Nr. 02}
\deadline{07.11.2018}

\begin{document}
\maketitle
\makeGradingTable[14]

% ╔══════════════════════════════════════════════════════════════════╗
% ║                            Aufgabe 01                            ║
% ╚══════════════════════════════════════════════════════════════════╝
\newproblem[1]{5}
\task{Sei \(s: \N \rightarrow \N\). Sei \(M= (Q,Σ,Γ,B,q_0,\overline{q},δ)\)
eine TM, welche nur Bandzellen zwischen einschließlich Positionen 0
und \(s(n)−1\) besucht. Zeigen Sie: Wenn \(M\) auf einer Eingabe \(w\)
der Länge \(n\) hält, dann hält \(M\) auf \(w\) nach spätestens
\(\card{Q}\cdot\card{\Gamma}^{s(n)}\cdot s(n)+1\) Schritten.} 
\proof Sei eine Turingmaschine \(M\) wie oben gegeben.  Wir können mit
sicherheit sagen, dass diese Turingmaschine nicht hält, wenn eine
Konfiguration --- also die Kombination aus dem beschriebenen Band, dem
aktiven Zustand des Steuerautomaten und der Position des
Schreib--/Leselopf --- zwei mal erreicht wird, da ein
deterministischer Automat in zwei identischen Situationen \emph{ipso
  facto} keine unterschiedlichen Entscheidungen treffen kann.

Da alle Bandzellen zweischen einschließlich den Postionen 0 und
\(s(n)-1\) besucht werden, ist die Anzahl der besucheten Zellen
\(s(n)\)\footnote{Zelle 0 wird auch besucht}. Die Menge der
Konfigurationen ist nun gegeben durch:
\begin{gather*}
  Q\times(\Gamma^{s(n)})\times[0,s(n)-1]
\end{gather*}
Die mächtigkeit dieser Menge ist nun:
\begin{gather*}
  \card{Q\times(\Gamma^{s(n)})\times[0,s(n)-1]}\\
  =\card{Q\times(\Gamma^{s(n)})}\cdot
  s(n)\\
  =\card{Q}\cdot\card{\Gamma^{s(n)}}\cdot
  s(n)\\
  = \card{Q} \cdot \card{\Gamma}^{s(n)} \cdot s(n)
\end{gather*}

Also kann eine Turingmaschine die auf einer Eingabe der länge \(n\)
hält höchstens \(\card{Q} \cdot \card{\Gamma}^{s(n)} \cdot s(n)\)
Schritte gehen, bevor sie den Ednzustand erreicht.  Sie terminiert
alsi nach spätestens \(\card{Q} \cdot \card{\Gamma}^{s(n)} \cdot s(n)
+1\) Schritten.\QED
% ╔══════════════════════════════════════════════════════════════════╗
% ║                            Aufgabe 02                            ║
% ╚══════════════════════════════════════════════════════════════════╝
\newproblem[2]{4}
\task{Sei \(L=\{w \# \bitneg{w} \vert w\in\{0, 1\}\}\) (über dem
  Alphabet \(\Sigma = \{0,1,\#\}\)), wobei \(\bitneg{w}\) die bitweise
  Negation von \(w\) ist (zB. \(\bitneg{\texttt{1011}} =
  \texttt{0100}\)). Beschreiben Sie eine möglichst effiziente
  2--Band--TM, die \(L\) entscheidet. Analysieren Sie den Zeit- und
  den Speicherplatzbedarf der von Ihnen entworfenen Maschine.} 
\solution
Die 2--Band--Turingmaschine kann \(L\) Zeiteffizienter als eine
1--Band--Turingmaschine entscheiden:
\begin{enumerate}
\item Terminiere mit \textsf{accept}, wenn ein Blank gelesen wird,
  ansonsten führe 2 aus, ohne Kopfpositionenn oder Bandbelegungen zu
  ändern.
\item Wenn auf Band 1 \texttt{0} oder \texttt{1} gelesen wird,
  schreibe dies auf Band 2; bewege beide Bänder nach rechts und bleibe in
  diesem Zustand, Wenn auf Band 1 \texttt{\#} gelesen wird, bewege
  Band 1 nach rechts, und auf Band 2 nach links, ändere beide
  Bandbelegungen nicht und gehe in Zustand 2. Wenn ein Blank gelesen
  wird, terminere mit \textsf{reject}
\item Ändere Kopfpostion und Bandbelegung auf Band 1 nicht, gehe auf
  Band 2 nach links, falls das auf Band 2 gelesene Zeichen \texttt{1}
  oder \texttt{0} ist, verwerfe falls es \texttt{\#} ist\footnote{Kann
    zwar nicht passieren, aber Vor-- ist ja besser als Nachsicht} und
  gehe nach rechts und in Zustand 3, falls ein Blank auf Band 2
  gelesen wird
\item Terminiere mit \texttt{accept}, falls auf beiden Bändern ein
  Blank gelesen wird. Gehe ohne zu Terminieren auf beiden Bändern nach
  rechts, falls auf den Bändern 1 und 2 zusammen genau ein mal
  \texttt{0} und \texttt{1} gelesen werden\footnote{Also die beiden
    Köpfe auf zwei unterschiedliche Symbole aus \{0,1\}
    stehen}. Terminere mit \textsf{reject} sonst.
\end{enumerate}

\keyword{Komplexitätsanalyse}
Sei \(w_1,w_2,w\in {0,1,\#}^*\) mit \(w=w_1\#w_2\) wenn \(w\in L\)
oder \(w=w_1w_2\) sonst. Seien
ferner \(n,n_1,n_2\in \N_0\) mit \(\card{w}=n, \card{w_1=n_1}\) und
  \(\card{w_2}=n_2\). 

Wenn \(w\) ein Wort aus \(L\) ist, Läuft die Turingmaschine über die
erste Hälfte des Eingabewortes (bis zum Doppelkreuz) und schreibt
dabei das gelesene auf das andere Band, danach läuft sie diese Strecke
auf Band 2 wieder zurück, ohne auf Band 1 die Postion zu
ändern. Danach läuft sie auf beiden Bändern bis zum Ende (also nochmal
\(\floor{\frac{n}{2}}\) schritte). Da in jedem Schritt eine Bewegung
auf dem Band 2 gemacht wird, lässt sich aus der Tatsache, dass jede
der \(\frac{n}{2}\) beschriebenen Zellen dieses Bandes genau 3 mal
Besucht werden herleiten, dass sowohl Laufzeit- als auch
Speicherkomplexität in diesem Fall in \(O(n)\) ist.

Falls das Wort kein Doppelkreuz besitzt, kopiert die Turingmaschine
alle \(n\) Zeichen von Band 1 auf Band 2, bevor sie mit \textsf{reject}
Terminiert, die Laufzeit-- und Speicherkomplexität ist also wiederum
jeweils \(O(n)\)


Wenn das Wort zwar ein Doppelkreuz besitzt, aber aus anderen Gründen
nicht in der Sprache ist, kopiert die TM wieder alle \(w_1\) Zeichen des
ersten Teilwortes auf Band 2, bevor sie dort zurück zum Anfang läuft
um anschließend auf beiden Bändern wieder nach rechts zu Laufen und
Terminiert nach spätestens \(w_2\) Schritten mit
\textsf{reject}. Die TM benötigt also \(w_1\cdot2+w_2\) Schritte; Laufzeit--
und Speicherkomplexität liegen also wieder in \(O(n)\).
% ╔══════════════════════════════════════════════════════════════════╗
% ║                            Aufgabe 03                            ║
% ╚══════════════════════════════════════════════════════════════════╝
\newproblem[3]{5}
\task{Eine TM mit einseitig unendlichem Band ist eine TM, die die
  Positionen \(p < 0\) nie benutzt. Zeigen Sie, dass jede 1-Band-TM
  (mit Akzeptieren oder Verwerfen als Ausgabe) durch eine 1-Band-TM
  mit einseitig unendlichem Band simuliert werden kann. Geben Sie ein
  möglichst effiziente Simulation. Wie groß ist der Zeitverlust Ihrer
  Simulation?} 

Sei \(M = (Q, Σ, Γ, B, q_0 , \bar q, δ)\) welche \(n \in
\N_0\cup\{\infty\}\) beliebige Bandzellen besucht.
Dann gibt es eine 1-Band-TM mit
einseitig unendlichem Band, welche die Turingmaschine \(M\)
simuliert. Diese TM markiert mit einem Zeichen, das nicht in
\(Γ\) ist, die erste Zelle des Bandes. Wenn diese Zelle besucht
und oder beschrieben werden soll verschiebt die TM das bisher auf dem
Band vorhandene Wort um eine Stelle nach rechts. Dies kann bei einer
TM mit endlichem Alphabet gemacht werden, indem die TM sich über
verschiedene Zustände merkt, welches Zeichen auf der aktuellen
Position ist und dieses mit dem der vorherigen ersetzt. Dann bewegt
sich die TM einen schritt weiter nach rechts und wiederholt den
Vorgang. Wenn der Bandinhalt um eine Stelle nach rechts verschoben
wurde, kann die TM wieder zurück an die zweite Bandzelle gehen und
dort normal weiterarbeiten oder noch das Zeichen einfügen und dann
weiterarbeiten. Der Kontext der TM wird dabei nicht verletzt, da die
Kopfposition relativ zu den anderen Informationen bestehen bleibt. Das
verschieben des gesamten Bandinhalts und das zurückgehen auf die erste
Zelle dauert für \(m \in \N\) beschriebene Zellen \(2m\). Bei einer
TM, die insgesamt \(p\in\N\) Zellen besucht können im Worst-Case alle
Zellen links des Bandbegins liegen, es ergibt sich also im Worst-Case
eine Verlangsamung der Laufzeit der Simulation um \(n!\) Schritte. 
\end{document}
