\documentclass[a4paper,twoside,subproblemsty=arabic,subsubproblemsty=alph, listings]{HomeworkAssignment}[2017/05/13 v2]

\usepackage[latin1]{inputenc}
\usepackage[english]{babel}
\usepackage{showframe} %Displays frames
\author{Max Mustermann Matr.Nr. 1701 \\ Autor Zwei Matr.Nr. 4711 \\ Dritter Kollaborateur Matr.Nr. 4242 }
\date{\today}
\abgabe{04.05.2017}
\tutorium{\"Ubungsgruppe 8}
\kurs{Betriebssysteme und Softwaretechnik}
\sheetTitle{Sample Sheet}
\begin{document}
\maketitle
\makeGradingTable
\newproblem[1]
\newproblem*[3]
\newproblem[2]

\newsubproblem[1]
\newsubsubproblem
\begin{lstlisting}[language=bash]
  $ uname -a
   Linux BuS1337 4.9.16-gentoo #1 SMP Fri Apr 28 16:18:12
   CEST 2017 x86_64 Intel(R) Core(TM) i7-6600U CPU @ 2.60GHz
   GenuineIntel GNU/Linux  
\end{lstlisting}

\begin{lstlisting}[language=bash]
  $ cat /etc/issue

  This is \n.\O (\s \m \r) \t
\end{lstlisting}
\begin{lstlisting}[language=bash]
  $ gcc --version
   gcc (Gentoo Hardened 5.4.0-r3 p1.3, pie-0.6.5) 5.4.0
   Copyright (C) 2015 Free Software Foundation, Inc.
   This is free software; see the source for copying
   conditions. There is NO warranty; not even for
   MERCHANTABILITY or FITNESS FOR A PARTICULAR PURPOSE.
\end{lstlisting}

\begin{lstlisting}[language=bash]
  $ w
   21:53:q25 up  3:36,  4 users,  load average: 0.00, 0.02, 0.09
   USER     TTY        LOGIN@   IDLE   JCPU   PCPU WHAT
   root     tty2      18:45   11:57   0.31s  0.31s -bash
   root     tty1      18:25    2:31m  0.60s  0.33s links
   georg    pts/0     19:29    1:59m  0.03s  0.03s -bash
   georg    pts/1     20:01    1.00s  0.07s  0.00s w
\end{lstlisting}
\newsubsubproblem

Aus \texttt{man man}:
\begin{lstlisting}[language=bash]
  man -k printf
           Search the short descriptions and manual page names for the
           keyword printf  as  regular expression.  Print out any
           matches. Equivalent to apropos printf.
\end{lstlisting}

Aus \texttt{man grep}:
\begin{lstlisting}[language=bash]
  -n, --line-number
            Prefix each line of output with the 1-based line number
            within its input file.
\end{lstlisting}

Aus \texttt{man fortune}:
\begin{lstlisting}[language=bash]
  HISTORY
       This  version of fortune is based on the NetBSD fortune 1.4,
       but with a number of bug fixes and enhancements.
\end{lstlisting}

\newsubproblem[2]
\newsubsubproblem
\begin{lstlisting}[language=bash]
   echo "Bus 2016: Abgabe der 1. Uebung am 6.5." | sed s/6/7/
  Bus 2017: Abgabe der 1. Uebung am 6.5.
\end{lstlisting}

\newsubsubproblem
Der Befehl \texttt{cut -d � � -f 1 d*} gibt jeweils aus allen
Dateien im aktuellen Ordner, die mit \texttt{d} beginnen den beginn
jeder Zeile bis zum ersten Leerzeichen aus. \texttt{-d � �} setzt das
Trennzeichen auf ein Leerzeichen. \texttt{-f 1} legt fest, dass nur
das erste Feld was mit diesem Trennzeichen gefunden wurde ausgegeben
wird. Durch \texttt{d*} wird der Befehl auf alle Dateien angewendet,
deren Namen mit d beginnt.

\newsubsubproblem
\begin{lstlisting}
  $ grep -B 19 -ne '^[0-9]\{5\}\ [a-zA-Z]\{1,\} [a-zA-Z]\{1,\}$' emails
  19017-Message-ID: <32509bccc5b1a43c@posteo.de>
  19018-Date: Tue, 26 Oct 2010 15:26:51 +0200
  19019-From: Arthur Dent <realArthurDent@posteo.de>
  ...
  19022-To: Emily Saunders <emily.saunders@mostlyharmless.com>
  ...
  19034-Arthur Dent
  19035-Galaxy 7
  19036:74369 Third Orbit
\end{lstlisting}
Die durch ``...'' markierten Teile der Ausgabe wurde aufgrund
von fehlender Relevanz f\"ur die Aufgabenstellung ausgelassen



\newsubproblem[3] %AUFGABE 3 =======================================
\newsubsubproblem

Der erste aufruf von \texttt{tr} im Befehl
\begin{lstlisting}
  $ tr -d '"?.!:;,+&'"'" < wotw.txt | tr -s " "
\end{lstlisting}%$
entfernt alle der spezifizierten Sonderzeichen aus der eingabe, welche
aus der Datei \texttt{wotw.txt} gepiped wird. Die Ausgabe des Befehles
wird in den zweiten \texttt{tr} aufruf gepiped, in welchen (durch das
flag \texttt{-s}) jedes doppelte Leerzeichen entfernt wird.
%\lstinputlisting{a3-1.txt}

\newsubsubproblem
Durch den Befehl
\begin{lstlisting}[language=bash]
  $ grep "road" -i -v -c wotw.txt
  6330
\end{lstlisting}%$
erf�hrt man, dass exakt 6330 Zeilen des Dokumentes das Wort ``road''
nicht enth�lt. Das flag \texttt{-i} sorgt daf�r, dass \texttt{grep} die
Gro�-/Kleinschreibung ignoriert, \texttt{-v} sorgt daf�r, dass nur
Zeilen ohne Match ausgegeben werden, \texttt{-c} Z\"ahlt die von
\texttt{grep} ausgegebenen Zeilen.
\newsubsubproblem
Der \texttt{grep} Befehl wandelt zun�chst die Datei in eine Liste
ihrer W�rter um. Danach werden die W�rter mit \texttt{sort} gruppiert
damit sie mit \texttt{uniq -c} zusammengefasst werden k�nnen und die
Anzahl ihrer Vorkommnisse bestimmt werden kann. Um die h�ufigsten 10
zu ermitteln wird dann mit \texttt{sort -n -r} nach den von
\texttt{uniq} hinzugef�gten Anzahlen absteigend sotiert und
schlussendlich werden mit \texttt{head -n 10} nur die ersten 10 ausgegeben.
\begin{lstlisting}[language=bash]
  grep -o '\<[[:alpha:]]*\>' wotw.txt | sort | uniq -c  | sort -n -r | head -n 10
  4417 the
  2373 and
  2284 of
  1554 a
  1300 I
  1160 to
  924 in
  853 was
  754 that
  568 had
\end{lstlisting}
\newsubproblem[4]
\newsubsubproblem
\begin{lstlisting}[language=bash]
  $ git init
\end{lstlisting}
\newsubsubproblem
\begin{lstlisting}[language=bash]
  $ git add A.txt B.txt C.txt
  $ git commit -m "Initial"
\end{lstlisting}
\newsubsubproblem
\begin{lstlisting}[language=bash]
  $ git diff
\end{lstlisting}
Gibt die alle �nderungen seit dem letzten commit aus. 
\begin{lstlisting}[language=bash]
  $ git diff $FILE
\end{lstlisting}
Vergleicht die Datei \texttt{\${}FILE} mit der Version aus dem letzen
Commit, und gibt alle Unterschiede aus.

\newsubsubproblem
\begin{lstlisting}[language=bash]
  $ git commit -m "Change files"      # commit
  $ git log                           # Anzeigen der Commits
\end{lstlisting}

\newsubproblem
\newsubsubproblem
\begin{lstlisting}[language=bash]
  $ convert B.png C.jpg BC.pdf && pdfunite A.pdf BC.pdf ABC.pdf
\end{lstlisting}

\newsubsubproblem
\begin{lstlisting}[language=bash]
  $ pdftk A.pdf cat 5-9 23 240-242 output Relevant.pdf
\end{lstlisting}


\end{document}