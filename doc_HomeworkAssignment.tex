\documentclass[a4papr]{article}
\usepackage[ngerman]{babel}
\usepackage[german=guillemets]{csquotes}

\title{Dokumentation zu HomeworkAssignment.cls}
\author{A. C. Hinrichs}

\begin{document}
\maketitle
\section{Befehle}
\begin{itemize}
\item[\texttt{author\{\#1\}}] Setzt den Autor des Dokumentes. Wie bei
  \textit{article}
\item[\texttt{date\{\#1\}}] Setzt das Datum des Dokumentes. Standard
  ist \texttt{\\today}
\item[\texttt{abgabe\{\#1\}}] Setzt das Abgabedatum. Erscheint auf der
  Titelseite oben Links.
\item[\texttt{tutorium\{\#1\}}] Setzt das Turtoriom bzw. die
  Kleingruppen\"ubung.
\item[\texttt{kurs\{\#1\}}] Setzt das Fach der Hausaufgabe.
\item[\texttt{problem\{\#1\}}] Startet eine neue Aufgabe mit \#1 als
  \"Uberschrift.  Die Aufgabe wird nicht nummeriert
\item[\texttt{newproblem[?1]}] Startet eine neue Aufgabe. Die
  \"Uberschrift ist \enquote{Aufgabe \texttt{\#Aufgaben}} mit
  \texttt{\#Aufgaben} als fortlaufenden Counter \"uber alle Aufgaben.
  Das optionale Argument \texttt{?1} ist die Nummer der Aufgabe, wird
  es \"ubergeben, so wird der Counter auf \texttt{?1} gesetzt.
\item[\texttt{subproblem\{\#1\}}] Startet eine neue Teilaufgabe mit
  \#1 als \"Uberschrift.  Die Teilufgabe wird nicht nummeriert
\item[\texttt{newsubproblem[?1]}] Startet eine neue Aufgabe. Die
  \"Uberschrift ist \enquote{Aufgabe
    \texttt{\#Aufgaben}.\texttt{\#Teilaufgaben}} mit
  \texttt{\#Teilaufgaben} als Buchstaben.  Das optionale Argument
  \texttt{?1} ist die Nummer der Teilaufgabe als Zahl.
\item[\texttt{subsubproblem\{\#1\}}] Startet eine neue
  Unterunteraufgabe mit \#1 als \"Uberschrift.  Die Teilufgabe wird
  nicht nummeriert
\item[\texttt{newsubsubproblem[?1]}] Startet eine neue
  Unterunteraufgabe. Die \"Uberschrift ist
  \enquote{\texttt{\#Unteraufgabe}} mit \texttt{\#Unteraufgabe} als
  kleine R\"omische Nummer.  Das optionale Argument \texttt{?1} ist
  die nummer der Teilaufgabe als Zahl.
\item[\texttt{solution[?1]}] Markiert den Anfang einer L\"osung.
  Wird ein argument \"ubergeben, so erzeugt dieses kommando
  \enquote{L\"osung \texttt{?1}:} ansonsten nur \enquote{L\"osung:}.
\item[\texttt{proof[?1]}] Markiert den Anfang eines Beweises.  Wird
  ein argument \"ubergeben, so erzeugt dieses kommando \enquote{Beweis
    \texttt{?1}:} ansonsten nur \enquote{Beweis:}.
\end{itemize}
\end{document}
