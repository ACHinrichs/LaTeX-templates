\documentclass[a4papr]{article}
\usepackage[ngerman]{babel}
\usepackage{csquotes}

\title{Dokumentation zu HomeworkAssignment.cls}
\author{A. C. Hinrichs}

\begin{document}
\maketitle
\section{Befehle}
\begin{itemize}
 \item[\texttt{author\{\#1\}}]		Setzt den Autor des Dokumentes. Wie bei \textit{article}
 \item[\texttt{date\{\#1\}}]		Setzt das Datum des Dokumentes. Standard ist \texttt{\\today}
 \item[\texttt{abgabe\{\#1\}}]		Setzt das Abgabedatum. Erscheint auf der Titelseite oben Links.
 \item[\texttt{tutorium\{\#1\}}]	Setzt das Turtoriom bzw. die Kleingruppen\"ubung.	
 \item[\texttt{problem\{\#1\}}] 	Startet eine neue Aufgabe mit \#1 als \"Uberschrift.  Die Aufgabe wird nicht nummeriert
 \item[\texttt{newproblem}] 		Startet eine neue Aufgabe. Die \"Uberschrift ist \enquote{Aufgabe \texttt{\#Aufgaben}} mit \texttt{\#Aufgaben} als fortlaufenden Counter \"uber alle Aufgaben
 \item[\texttt{subproblem\{\#1\}}] 	Startet eine neue Teilaufgabe mit \#1 als \"Uberschrift.  Die Teilufgabe wird nicht nummeriert
 \item[\texttt{newsubproblem}] 		Startet eine neue Aufgabe. Die \"Uberschrift ist \enquote{Aufgabe \texttt{\#Aufgaben}.\texttt{\#Teilaufgaben}} mit \texttt{\#Teilaufgaben} als Buchstaben.
 \item[\texttt{subsubproblem\{\#1\}}] 	Startet eine neue Unterunteraufgabe mit \#1 als \"Uberschrift.  Die Teilufgabe wird nicht nummeriert
 \item[\texttt{newsubsubproblem}] 	Startet eine neue Unterunteraufgabe. Die \"Uberschrift ist \enquote{\texttt{\#Unteraufgabe}} mit \texttt{\#Unteraufgabe} als kleine R\"omische Nummer.
\end{itemize}
\end{document}
